% Options for packages loaded elsewhere
% Options for packages loaded elsewhere
\PassOptionsToPackage{unicode}{hyperref}
\PassOptionsToPackage{hyphens}{url}
\PassOptionsToPackage{dvipsnames,svgnames,x11names}{xcolor}
%
\documentclass[
  letterpaper,
  DIV=11,
  numbers=noendperiod]{scrartcl}
\usepackage{xcolor}
\usepackage{amsmath,amssymb}
\setcounter{secnumdepth}{-\maxdimen} % remove section numbering
\usepackage{iftex}
\ifPDFTeX
  \usepackage[T1]{fontenc}
  \usepackage[utf8]{inputenc}
  \usepackage{textcomp} % provide euro and other symbols
\else % if luatex or xetex
  \usepackage{unicode-math} % this also loads fontspec
  \defaultfontfeatures{Scale=MatchLowercase}
  \defaultfontfeatures[\rmfamily]{Ligatures=TeX,Scale=1}
\fi
\usepackage{lmodern}
\ifPDFTeX\else
  % xetex/luatex font selection
\fi
% Use upquote if available, for straight quotes in verbatim environments
\IfFileExists{upquote.sty}{\usepackage{upquote}}{}
\IfFileExists{microtype.sty}{% use microtype if available
  \usepackage[]{microtype}
  \UseMicrotypeSet[protrusion]{basicmath} % disable protrusion for tt fonts
}{}
\makeatletter
\@ifundefined{KOMAClassName}{% if non-KOMA class
  \IfFileExists{parskip.sty}{%
    \usepackage{parskip}
  }{% else
    \setlength{\parindent}{0pt}
    \setlength{\parskip}{6pt plus 2pt minus 1pt}}
}{% if KOMA class
  \KOMAoptions{parskip=half}}
\makeatother
% Make \paragraph and \subparagraph free-standing
\makeatletter
\ifx\paragraph\undefined\else
  \let\oldparagraph\paragraph
  \renewcommand{\paragraph}{
    \@ifstar
      \xxxParagraphStar
      \xxxParagraphNoStar
  }
  \newcommand{\xxxParagraphStar}[1]{\oldparagraph*{#1}\mbox{}}
  \newcommand{\xxxParagraphNoStar}[1]{\oldparagraph{#1}\mbox{}}
\fi
\ifx\subparagraph\undefined\else
  \let\oldsubparagraph\subparagraph
  \renewcommand{\subparagraph}{
    \@ifstar
      \xxxSubParagraphStar
      \xxxSubParagraphNoStar
  }
  \newcommand{\xxxSubParagraphStar}[1]{\oldsubparagraph*{#1}\mbox{}}
  \newcommand{\xxxSubParagraphNoStar}[1]{\oldsubparagraph{#1}\mbox{}}
\fi
\makeatother

\usepackage{color}
\usepackage{fancyvrb}
\newcommand{\VerbBar}{|}
\newcommand{\VERB}{\Verb[commandchars=\\\{\}]}
\DefineVerbatimEnvironment{Highlighting}{Verbatim}{commandchars=\\\{\}}
% Add ',fontsize=\small' for more characters per line
\usepackage{framed}
\definecolor{shadecolor}{RGB}{241,243,245}
\newenvironment{Shaded}{\begin{snugshade}}{\end{snugshade}}
\newcommand{\AlertTok}[1]{\textcolor[rgb]{0.68,0.00,0.00}{#1}}
\newcommand{\AnnotationTok}[1]{\textcolor[rgb]{0.37,0.37,0.37}{#1}}
\newcommand{\AttributeTok}[1]{\textcolor[rgb]{0.40,0.45,0.13}{#1}}
\newcommand{\BaseNTok}[1]{\textcolor[rgb]{0.68,0.00,0.00}{#1}}
\newcommand{\BuiltInTok}[1]{\textcolor[rgb]{0.00,0.23,0.31}{#1}}
\newcommand{\CharTok}[1]{\textcolor[rgb]{0.13,0.47,0.30}{#1}}
\newcommand{\CommentTok}[1]{\textcolor[rgb]{0.37,0.37,0.37}{#1}}
\newcommand{\CommentVarTok}[1]{\textcolor[rgb]{0.37,0.37,0.37}{\textit{#1}}}
\newcommand{\ConstantTok}[1]{\textcolor[rgb]{0.56,0.35,0.01}{#1}}
\newcommand{\ControlFlowTok}[1]{\textcolor[rgb]{0.00,0.23,0.31}{\textbf{#1}}}
\newcommand{\DataTypeTok}[1]{\textcolor[rgb]{0.68,0.00,0.00}{#1}}
\newcommand{\DecValTok}[1]{\textcolor[rgb]{0.68,0.00,0.00}{#1}}
\newcommand{\DocumentationTok}[1]{\textcolor[rgb]{0.37,0.37,0.37}{\textit{#1}}}
\newcommand{\ErrorTok}[1]{\textcolor[rgb]{0.68,0.00,0.00}{#1}}
\newcommand{\ExtensionTok}[1]{\textcolor[rgb]{0.00,0.23,0.31}{#1}}
\newcommand{\FloatTok}[1]{\textcolor[rgb]{0.68,0.00,0.00}{#1}}
\newcommand{\FunctionTok}[1]{\textcolor[rgb]{0.28,0.35,0.67}{#1}}
\newcommand{\ImportTok}[1]{\textcolor[rgb]{0.00,0.46,0.62}{#1}}
\newcommand{\InformationTok}[1]{\textcolor[rgb]{0.37,0.37,0.37}{#1}}
\newcommand{\KeywordTok}[1]{\textcolor[rgb]{0.00,0.23,0.31}{\textbf{#1}}}
\newcommand{\NormalTok}[1]{\textcolor[rgb]{0.00,0.23,0.31}{#1}}
\newcommand{\OperatorTok}[1]{\textcolor[rgb]{0.37,0.37,0.37}{#1}}
\newcommand{\OtherTok}[1]{\textcolor[rgb]{0.00,0.23,0.31}{#1}}
\newcommand{\PreprocessorTok}[1]{\textcolor[rgb]{0.68,0.00,0.00}{#1}}
\newcommand{\RegionMarkerTok}[1]{\textcolor[rgb]{0.00,0.23,0.31}{#1}}
\newcommand{\SpecialCharTok}[1]{\textcolor[rgb]{0.37,0.37,0.37}{#1}}
\newcommand{\SpecialStringTok}[1]{\textcolor[rgb]{0.13,0.47,0.30}{#1}}
\newcommand{\StringTok}[1]{\textcolor[rgb]{0.13,0.47,0.30}{#1}}
\newcommand{\VariableTok}[1]{\textcolor[rgb]{0.07,0.07,0.07}{#1}}
\newcommand{\VerbatimStringTok}[1]{\textcolor[rgb]{0.13,0.47,0.30}{#1}}
\newcommand{\WarningTok}[1]{\textcolor[rgb]{0.37,0.37,0.37}{\textit{#1}}}

\usepackage{longtable,booktabs,array}
\usepackage{calc} % for calculating minipage widths
% Correct order of tables after \paragraph or \subparagraph
\usepackage{etoolbox}
\makeatletter
\patchcmd\longtable{\par}{\if@noskipsec\mbox{}\fi\par}{}{}
\makeatother
% Allow footnotes in longtable head/foot
\IfFileExists{footnotehyper.sty}{\usepackage{footnotehyper}}{\usepackage{footnote}}
\makesavenoteenv{longtable}
\usepackage{graphicx}
\makeatletter
\newsavebox\pandoc@box
\newcommand*\pandocbounded[1]{% scales image to fit in text height/width
  \sbox\pandoc@box{#1}%
  \Gscale@div\@tempa{\textheight}{\dimexpr\ht\pandoc@box+\dp\pandoc@box\relax}%
  \Gscale@div\@tempb{\linewidth}{\wd\pandoc@box}%
  \ifdim\@tempb\p@<\@tempa\p@\let\@tempa\@tempb\fi% select the smaller of both
  \ifdim\@tempa\p@<\p@\scalebox{\@tempa}{\usebox\pandoc@box}%
  \else\usebox{\pandoc@box}%
  \fi%
}
% Set default figure placement to htbp
\def\fps@figure{htbp}
\makeatother





\setlength{\emergencystretch}{3em} % prevent overfull lines

\providecommand{\tightlist}{%
  \setlength{\itemsep}{0pt}\setlength{\parskip}{0pt}}



 


\KOMAoption{captions}{tableheading}
\makeatletter
\@ifpackageloaded{caption}{}{\usepackage{caption}}
\AtBeginDocument{%
\ifdefined\contentsname
  \renewcommand*\contentsname{Table of contents}
\else
  \newcommand\contentsname{Table of contents}
\fi
\ifdefined\listfigurename
  \renewcommand*\listfigurename{List of Figures}
\else
  \newcommand\listfigurename{List of Figures}
\fi
\ifdefined\listtablename
  \renewcommand*\listtablename{List of Tables}
\else
  \newcommand\listtablename{List of Tables}
\fi
\ifdefined\figurename
  \renewcommand*\figurename{Figure}
\else
  \newcommand\figurename{Figure}
\fi
\ifdefined\tablename
  \renewcommand*\tablename{Table}
\else
  \newcommand\tablename{Table}
\fi
}
\@ifpackageloaded{float}{}{\usepackage{float}}
\floatstyle{ruled}
\@ifundefined{c@chapter}{\newfloat{codelisting}{h}{lop}}{\newfloat{codelisting}{h}{lop}[chapter]}
\floatname{codelisting}{Listing}
\newcommand*\listoflistings{\listof{codelisting}{List of Listings}}
\makeatother
\makeatletter
\makeatother
\makeatletter
\@ifpackageloaded{caption}{}{\usepackage{caption}}
\@ifpackageloaded{subcaption}{}{\usepackage{subcaption}}
\makeatother
\usepackage{bookmark}
\IfFileExists{xurl.sty}{\usepackage{xurl}}{} % add URL line breaks if available
\urlstyle{same}
\hypersetup{
  pdftitle={Johnson and Shot Noise Analysis},
  pdfauthor={Nathan Chien},
  colorlinks=true,
  linkcolor={blue},
  filecolor={Maroon},
  citecolor={Blue},
  urlcolor={Blue},
  pdfcreator={LaTeX via pandoc}}


\title{Johnson and Shot Noise Analysis}
\author{Nathan Chien}
\date{}
\begin{document}
\maketitle

\renewcommand*\contentsname{Table of contents}
{
\hypersetup{linkcolor=}
\setcounter{tocdepth}{3}
\tableofcontents
}

\section{Johnson and Shot Noise Analysis
(2024)}\label{johnson-and-shot-noise-analysis-2024}

\paragraph{This template assumes use of the new shot noise
source}\label{this-template-assumes-use-of-the-new-shot-noise-source}

Use this template to carry out the analysis tasks for the Noise
experiment. For reference, here are links to recommended Python
resources: the
\href{https://jakevdp.github.io/WhirlwindTourOfPython/}{Whirlwind Tour
of Python} and the
\href{https://jakevdp.github.io/PythonDataScienceHandbook/}{Python Data
Science Handbook} both by Jake VanderPlas. \#\# First, import some
packages

This is a good idea at the beginning of your notebook to include the
packages that you will need. We will use those shown below here. A brief
description: * \texttt{numpy} is the foundational package for Python
numerical work. It extends and speeds up array operations beyond
standard Python, and it includes almost all math functions that you
would need for example \texttt{sqrt()} (square root) or \texttt{cos()}
(cosine). These would be written in code as \texttt{np.sqrt()} or
\texttt{np.cos()}. * \texttt{scipy} is a huge collection of scientific
data analysis functions, routines, physicical constants, etc. This is
the second most used package for scientific work. Here we will use the
physical constants library, \texttt{scipy.constants}. Documentation is
at \href{https://docs.scipy.org/doc/scipy/reference/}{SciPy.org} with
the constants subpackage at
https://docs.scipy.org/doc/scipy/reference/constants.html. *
\texttt{uncertainties} is a very useful small package that simplifies
uncertainty propagation and printing out of quantities with uncertainty.
Documentation is at https://pythonhosted.org/uncertainties/ *
\texttt{matplotlib} is \emph{the} standard plotting package for
scientific Python. We will use a subset called \texttt{pyplot} which is
modeled after the plotting functions used in MATLAB. The last line
below, \texttt{\%matplotlib\ inline}, simply forces the plots to appear
within the notebook. * \texttt{pandas} is a large data science package.
It's main feature is a set of methods to create and manipulate a
``DataFrame,'' which is an enlargement of the idea of an array. I plays
well with NumPy and other packages. We will use it mainly as a way to
read files into data sets in an easy way.

We will also use the \href{https://lmfit.github.io/lmfit-py/}{LMFit}
package to make line fits. This will be explained later in the notebook.

\begin{Shaded}
\begin{Highlighting}[]
\ImportTok{from}\NormalTok{ json.decoder }\ImportTok{import}\NormalTok{ NaN}

\CommentTok{\# Run this cell with Shift{-}Enter, and wait until the}
\CommentTok{\# asterisk changes to a number, i.e., [*] becomes [1]}
\ImportTok{import}\NormalTok{ numpy }\ImportTok{as}\NormalTok{ np}
\ImportTok{import}\NormalTok{ pandas}
\ImportTok{import}\NormalTok{ scipy.constants }\ImportTok{as}\NormalTok{ const}
\ImportTok{import}\NormalTok{ uncertainties }\ImportTok{as}\NormalTok{ unc}
\ImportTok{import}\NormalTok{ matplotlib.pyplot }\ImportTok{as}\NormalTok{ plt}
\ImportTok{import}\NormalTok{ pandas }\ImportTok{as}\NormalTok{ pd}
\ImportTok{import}\NormalTok{ re }\ImportTok{as}\NormalTok{ re}
\ImportTok{import}\NormalTok{ mplcyberpunk}
\ImportTok{from}\NormalTok{ numpy.f2py.auxfuncs }\ImportTok{import}\NormalTok{ throw\_error}
\ImportTok{from}\NormalTok{ numpy.ma.core }\ImportTok{import}\NormalTok{ size}
\ImportTok{from}\NormalTok{ uncertainties }\ImportTok{import}\NormalTok{ ufloat}
\ImportTok{from}\NormalTok{ uncertainties.unumpy }\ImportTok{import}\NormalTok{ uarray}
\OperatorTok{\%}\NormalTok{matplotlib inline}

\NormalTok{plt.style.use(}\StringTok{\textquotesingle{}cyberpunk\textquotesingle{}}\NormalTok{)}
\end{Highlighting}
\end{Shaded}

\subsection{Johnson Noise Analysis}\label{johnson-noise-analysis}

\subsection{Exercise 1 - Data
reduction}\label{exercise-1---data-reduction}

\begin{quote}
\subsubsection{Read in the raw data}\label{read-in-the-raw-data}

\textbf{About Data Files:} This template assumes that the data files
will have one of two types of structure: 1. If you took 5 readings (or
so) for each measurement and plan to average them here, the assumed
structure is one where each spreadsheet column is named with the
resistance (for \textgreater{} Johnson Noise), e.g., ``9.99k'', or the
emission current (for shot noise), e.g., ``0.1047mA'' and each row of
the data is one trial with each cell containing the measured RMS voltage
\textgreater{} in the frequency band; 2. If you opted instead to simply
take one longer-average time measurement for each resistance (or
emission current), then the assumed structure would be two columns, the
first \textgreater{} column headed with resistance (or emission current)
and the second column headed with the measured RMS voltage in the
frequency band.

Below, these structures are treated by the designation ``1'' or ``2''.
Stucture type ``1'' requires a little more effort to reduce, but offers
the option of calculating data point \textgreater{} uncertainties.

\textbf{Advice:} Use the \textbf{Pandas} function \texttt{read\_csv()}
to pull the file into a Pandas Dataframe, like this:

\begin{verbatim}
johnson_294 = pd.read_csv('Johnson_294K.csv')
\end{verbatim}
\end{quote}

\begin{quote}
If the last line in the code cell is the name of the DataFrame
(\texttt{johnson\_294}), the notebook cell will print a nice table.
\end{quote}

\begin{quote}
You may obtain the arrays for each column by using the column label,
e.g.,
\texttt{johnson\_294{[}\textquotesingle{}40.0k\textquotesingle{}{]}} is
the array of the first column.
\end{quote}

For Johnson shot noise, we went about taking data manually and averaging
it.

\begin{Shaded}
\begin{Highlighting}[]
\CommentTok{\# This is to import the data as a raw pd dataframe.}

\NormalTok{johnson\_294 }\OperatorTok{=}\NormalTok{ pd.read\_csv(}\StringTok{\textquotesingle{}rawData/Johnson data {-} Johnson\_Room\_jupyter.csv\textquotesingle{}}\NormalTok{)}
\NormalTok{johnson\_294  }\CommentTok{\# DataFrame name on the last line spits out a table}
\end{Highlighting}
\end{Shaded}

\begin{longtable}[]{@{}llllllllll@{}}
\toprule\noalign{}
& 40k & 20k & 15k & 9.99k & 7.5k & 4.99k & 2.5k & 1.0k & 0.0K \\
\midrule\noalign{}
\endhead
\bottomrule\noalign{}
\endlastfoot
0 & 0.007909 & 0.005793 & 0.005050 & 0.004211 & 0.003701 & 0.002997 &
0.002196 & 0.001457 & 0.000620 \\
1 & 0.007962 & 0.005885 & 0.005047 & 0.004180 & 0.003633 & 0.003006 &
0.002207 & 0.001484 & 0.000624 \\
2 & 0.007796 & 0.005765 & 0.005031 & 0.004170 & 0.003611 & 0.002980 &
0.002148 & 0.001482 & 0.000615 \\
3 & 0.007801 & 0.005848 & 0.005163 & 0.004190 & 0.003652 & 0.003037 &
0.002192 & 0.001496 & 0.000629 \\
4 & 0.007821 & 0.005864 & 0.005107 & 0.004182 & 0.003747 & 0.002973 &
0.002194 & 0.001473 & 0.000625 \\
\end{longtable}

\subsubsection{Massage the raw data}\label{massage-the-raw-data}

\begin{quote}
\mbox{}%
\paragraph{For data structure type ``1.'' If you have a type ``2'' data
structure, skip to ``Plot the Reduced Data''
below.}\label{for-data-structure-type-1.-if-you-have-a-type-2-data-structure-skip-to-plot-the-reduced-data-below.}

Create new arrays that have averages of the 5 readings at each value of
the resistance and their standard deviation. Then extract the resistance
from the column label and make \textgreater{} into a number. Finally,
build a new DataFrame that has these arrays. Below is an example. The
example shows a number of useful operations. Study it carefully.

We will use a loop to build the new arrays first, and then combine them
into a DataFrame.

You can extract the resistance from the column heading. Here is one way
to do it, assuming \texttt{col\_label} is the column label:

\begin{verbatim}
resistance = float(col_label.split('k')[0])
\end{verbatim}

This splits the label at \texttt{k} and puts the number into the first
(0) position as a string. \texttt{float()} converts the number string to
a flaoting point number.

Then calculate the mean and standard deviation using the built-in
methods for the arrays.
\end{quote}

\begin{Shaded}
\begin{Highlighting}[]
\CommentTok{\#\# Study this example.}

\CommentTok{\# These lines create empty arrays that will be filled.}
\NormalTok{Rs }\OperatorTok{=}\NormalTok{ np.zeros(}\DecValTok{0}\NormalTok{)}
\NormalTok{Vs }\OperatorTok{=}\NormalTok{ np.zeros(}\DecValTok{0}\NormalTok{)}
\NormalTok{Stds }\OperatorTok{=}\NormalTok{ np.zeros(}\DecValTok{0}\NormalTok{)}

\CommentTok{\# This is a standard Python loop.  Note the \textquotesingle{}for \textless{}item\textgreater{} in \textless{}list\textgreater{}:\textquotesingle{} construction}
\CommentTok{\# The \textquotesingle{}.columns\textquotesingle{} is an array of column labels in the DataFrame}
\ControlFlowTok{for}\NormalTok{ label }\KeywordTok{in}\NormalTok{ johnson\_294.columns:}
    \CommentTok{\# obtain the numerical part of the column label}
    \CommentTok{\# R = float(label.split(\textquotesingle{}k\textquotesingle{})[0])}
\NormalTok{    R }\OperatorTok{=} \BuiltInTok{float}\NormalTok{(re.split(}\VerbatimStringTok{r"}\FunctionTok{(?i)}\VerbatimStringTok{k"}\NormalTok{, label)[}\DecValTok{0}\NormalTok{])}
    \CommentTok{\# calculate the mean (average) of the numbers in the column}
\NormalTok{    mean }\OperatorTok{=}\NormalTok{ johnson\_294[label].mean()}
    \CommentTok{\# calculate the standard deviation of the same numbers}
\NormalTok{    std }\OperatorTok{=}\NormalTok{ johnson\_294[label].std()}

    \CommentTok{\# These lines add each calculated result to the associated array}
\NormalTok{    Rs }\OperatorTok{=}\NormalTok{ np.append(Rs,R)}
\NormalTok{    Vs }\OperatorTok{=}\NormalTok{ np.append(Vs,mean)}
\NormalTok{    Stds }\OperatorTok{=}\NormalTok{ np.append(Stds,std)}

\NormalTok{Rs}

\CommentTok{\# Initialize an empty DataFrame}
\NormalTok{J\_294 }\OperatorTok{=}\NormalTok{ pd.DataFrame()}

\CommentTok{\# These lines add columns to the DataFrame}
\NormalTok{J\_294[}\StringTok{\textquotesingle{}R (ohms)\textquotesingle{}}\NormalTok{] }\OperatorTok{=}\NormalTok{ Rs}\OperatorTok{*}\FloatTok{1000.0} \CommentTok{\# Convert resistance from kohms to ohms}
\NormalTok{J\_294[}\StringTok{\textquotesingle{}Vrms (V)\textquotesingle{}}\NormalTok{] }\OperatorTok{=}\NormalTok{ Vs}
\NormalTok{J\_294[}\StringTok{\textquotesingle{}DV (V)\textquotesingle{}}\NormalTok{] }\OperatorTok{=}\NormalTok{ Stds}

\CommentTok{\# Here is another way to do the same thing}

\NormalTok{J\_294\_ver2 }\OperatorTok{=}\NormalTok{ pd.DataFrame(\{}
    \StringTok{\textquotesingle{}R (ohms)\textquotesingle{}}\NormalTok{: Rs}\OperatorTok{*}\FloatTok{1e3}\NormalTok{,}
    \StringTok{\textquotesingle{}Vrms (V)\textquotesingle{}}\NormalTok{: Vs,}
    \StringTok{\textquotesingle{}DV (V)\textquotesingle{}}\NormalTok{: Stds}
\NormalTok{\})}

\CommentTok{\# Finally display the results.  Name of DataFrame on last line spits a table}
\CommentTok{\# J\_294}
\NormalTok{J\_294\_ver2}
\end{Highlighting}
\end{Shaded}

\begin{longtable}[]{@{}llll@{}}
\toprule\noalign{}
& R (ohms) & Vrms (V) & DV (V) \\
\midrule\noalign{}
\endhead
\bottomrule\noalign{}
\endlastfoot
0 & 40000.0 & 0.007858 & 0.000074 \\
1 & 20000.0 & 0.005831 & 0.000050 \\
2 & 15000.0 & 0.005080 & 0.000055 \\
3 & 9990.0 & 0.004187 & 0.000015 \\
4 & 7500.0 & 0.003669 & 0.000055 \\
5 & 4990.0 & 0.002999 & 0.000025 \\
6 & 2500.0 & 0.002187 & 0.000023 \\
7 & 1000.0 & 0.001478 & 0.000015 \\
8 & 0.0 & 0.000623 & 0.000005 \\
\end{longtable}

\paragraph{Repeat for the other
temperature}\label{repeat-for-the-other-temperature}

\begin{quote}
Now you try it for the other temperature data set.
\end{quote}

\begin{Shaded}
\begin{Highlighting}[]
\CommentTok{\#\# Read in the data set and display it}
\NormalTok{johnson\_ln2 }\OperatorTok{=}\NormalTok{ pd.read\_csv(}\StringTok{"rawData/Johnson data {-} Johnson\_ln2\_jupyter.csv"}\NormalTok{)}
\NormalTok{johnson\_ln2}
\end{Highlighting}
\end{Shaded}

\begin{longtable}[]{@{}llllllllll@{}}
\toprule\noalign{}
& 40k & 20k & 15k & 9.99k & 7.5k & 4.99k & 2.5k & 1.0k & 0.0k \\
\midrule\noalign{}
\endhead
\bottomrule\noalign{}
\endlastfoot
0 & 0.004315 & 0.003218 & 0.002730 & 0.002270 & 0.002021 & 0.001781 &
0.001299 & 0.000932 & 0.000627 \\
1 & 0.004319 & 0.003168 & 0.002759 & 0.002272 & 0.001994 & 0.001776 &
0.001246 & 0.000927 & 0.000627 \\
2 & 0.004299 & 0.003138 & 0.002712 & 0.002264 & 0.001972 & 0.001746 &
0.001243 & 0.000947 & 0.000617 \\
3 & 0.004424 & 0.003201 & 0.002749 & 0.002202 & 0.002000 & 0.001729 &
0.001265 & 0.000941 & 0.000623 \\
4 & 0.004292 & 0.003161 & 0.002731 & 0.002289 & 0.002032 & 0.001765 &
0.001260 & 0.000944 & 0.000628 \\
5 & 0.004286 & 0.003098 & 0.002715 & 0.002254 & 0.001981 & 0.001732 &
0.001245 & 0.000929 & 0.000622 \\
\end{longtable}

\begin{Shaded}
\begin{Highlighting}[]
\CommentTok{\#\# Calculate the averages of each column.}
\CommentTok{\#\# extract the values of the resistance.  }
\CommentTok{\#\# Build a (new) dataframe and display it to see if it looks right.}
\CommentTok{\# These lines create empty arrays that will be filled.}
\NormalTok{Rs }\OperatorTok{=}\NormalTok{ np.zeros(}\DecValTok{0}\NormalTok{)}
\NormalTok{Vs }\OperatorTok{=}\NormalTok{ np.zeros(}\DecValTok{0}\NormalTok{)}
\NormalTok{Stds }\OperatorTok{=}\NormalTok{ np.zeros(}\DecValTok{0}\NormalTok{)}

\CommentTok{\# This is a standard Python loop.  Note the \textquotesingle{}for \textless{}item\textgreater{} in \textless{}list\textgreater{}:\textquotesingle{} construction}
\CommentTok{\# The \textquotesingle{}.columns\textquotesingle{} is an array of column labels in the DataFrame}
\ControlFlowTok{for}\NormalTok{ label }\KeywordTok{in}\NormalTok{ johnson\_ln2.columns:}
    \CommentTok{\# obtain the numerical part of the column label}
    \CommentTok{\# R = float(label.split(\textquotesingle{}k\textquotesingle{})[0])}
\NormalTok{    R }\OperatorTok{=} \BuiltInTok{float}\NormalTok{(re.split(}\VerbatimStringTok{r"}\FunctionTok{(?i)}\VerbatimStringTok{k"}\NormalTok{, label)[}\DecValTok{0}\NormalTok{])}
    \CommentTok{\# calculate the mean (average) of the numbers in the column}
\NormalTok{    mean }\OperatorTok{=}\NormalTok{ johnson\_ln2[label].mean()}
    \CommentTok{\# calculate the standard deviation of the same numbers}
\NormalTok{    std }\OperatorTok{=}\NormalTok{ johnson\_ln2[label].std()}

    \CommentTok{\# These lines add each calculated result to the associated array}
\NormalTok{    Rs }\OperatorTok{=}\NormalTok{ np.append(Rs,R)}
\NormalTok{    Vs }\OperatorTok{=}\NormalTok{ np.append(Vs,mean)}
\NormalTok{    Stds }\OperatorTok{=}\NormalTok{ np.append(Stds,std)}


\NormalTok{J\_ln2 }\OperatorTok{=}\NormalTok{ pd.DataFrame(\{}
    \StringTok{\textquotesingle{}R (ohms)\textquotesingle{}}\NormalTok{: Rs}\OperatorTok{*}\FloatTok{1e3}\NormalTok{,}
    \StringTok{\textquotesingle{}Vrms (V)\textquotesingle{}}\NormalTok{: Vs,}
    \StringTok{\textquotesingle{}DV (V)\textquotesingle{}}\NormalTok{: Stds}
\NormalTok{\})}

\NormalTok{J\_ln2}
\end{Highlighting}
\end{Shaded}

\begin{longtable}[]{@{}llll@{}}
\toprule\noalign{}
& R (ohms) & Vrms (V) & DV (V) \\
\midrule\noalign{}
\endhead
\bottomrule\noalign{}
\endlastfoot
0 & 40000.0 & 0.004322 & 0.000051 \\
1 & 20000.0 & 0.003164 & 0.000043 \\
2 & 15000.0 & 0.002733 & 0.000018 \\
3 & 9990.0 & 0.002258 & 0.000030 \\
4 & 7500.0 & 0.002000 & 0.000023 \\
5 & 4990.0 & 0.001755 & 0.000022 \\
6 & 2500.0 & 0.001260 & 0.000021 \\
7 & 1000.0 & 0.000937 & 0.000008 \\
8 & 0.0 & 0.000624 & 0.000004 \\
\end{longtable}

\subsubsection{Plot the reduced data}\label{plot-the-reduced-data}

\begin{quote}
Plot the data set of \(V_{rms}\) vs \(R\) to see what it looks like.

Below, I'll show how. Study the commands, change them, and see what
happens. Hint: study the sections in the
\href{https://jakevdp.github\%3E\%20.io/PythonDataScienceHandbook/}{Python
Data Science Handbook} on Matplotlib.

After you make the plot, always look to make sure your data set does not
have any weird points. This is a good way to catch bad data and/or
mistakes.
\end{quote}

\begin{Shaded}
\begin{Highlighting}[]
\CommentTok{\# Set up plot defaults  This cell only needs to be executed once.}
\ImportTok{import}\NormalTok{ matplotlib }\ImportTok{as}\NormalTok{ mpl}
\NormalTok{mpl.rcParams[}\StringTok{\textquotesingle{}figure.figsize\textquotesingle{}}\NormalTok{] }\OperatorTok{=} \FloatTok{10.0}\NormalTok{,}\FloatTok{8.0}  \CommentTok{\# Roughly 11 cm wde by 8 cm high}
\NormalTok{mpl.rcParams[}\StringTok{\textquotesingle{}font.size\textquotesingle{}}\NormalTok{] }\OperatorTok{=} \FloatTok{12.0} \CommentTok{\# Use 12 point font}
\end{Highlighting}
\end{Shaded}

\begin{Shaded}
\begin{Highlighting}[]
\CommentTok{\#\# Plot the data sets on one graph}
\CommentTok{\#\# Header commands provided}


\NormalTok{fig, ax }\OperatorTok{=}\NormalTok{ plt.subplots()}

\NormalTok{ax.set\_title(}\StringTok{\textquotesingle{}Johnson Noise Data\textquotesingle{}}\NormalTok{)}
\NormalTok{ax.set\_xlabel(}\VerbatimStringTok{r\textquotesingle{}Resistance }\DecValTok{$}\VerbatimStringTok{R}\DecValTok{$}\VerbatimStringTok{ }\KeywordTok{(}\DecValTok{$}\ErrorTok{\textbackslash{}}\VerbatimStringTok{Omega}\DecValTok{$}\KeywordTok{)}\VerbatimStringTok{\textquotesingle{}}\NormalTok{)}
\NormalTok{ax.set\_ylabel(}\VerbatimStringTok{r\textquotesingle{}}\DecValTok{$}\VerbatimStringTok{V\_J}\DecValTok{$}\VerbatimStringTok{ }\KeywordTok{(}\VerbatimStringTok{Vrms}\KeywordTok{)}\VerbatimStringTok{\textquotesingle{}}\NormalTok{)}
\NormalTok{ax.grid(color}\OperatorTok{=}\StringTok{\textquotesingle{}\#2A3459\textquotesingle{}}\NormalTok{)}



\CommentTok{\# plt.grid() \# Turn on the grid}
\CommentTok{\# plt.title(\textquotesingle{}Johnson Noise Data\textquotesingle{}) \# make a plot title}
\CommentTok{\# plt.ylabel(r\textquotesingle{}$V\_J$ (Vrms)\textquotesingle{}) \# Make an axis label.  Note the $$ to typeset math}
\CommentTok{\# plt.xlabel(r\textquotesingle{}Resistance $R$ ($\textbackslash{}Omega$)\textquotesingle{}) \#Another axis label}

\CommentTok{\# Below shows how to make a plot with error bars.  The errors are multiplied by }
\CommentTok{\# 10 so that the bars are visible. }

\CommentTok{\# If you have no error bars, simply omit the item \textquotesingle{}yerr=J\_294[\textquotesingle{}DV (V)\textquotesingle{}]*10\textquotesingle{}}

\NormalTok{ax.errorbar(J\_294[}\StringTok{\textquotesingle{}R (ohms)\textquotesingle{}}\NormalTok{],J\_294[}\StringTok{\textquotesingle{}Vrms (V)\textquotesingle{}}\NormalTok{],}
\NormalTok{             yerr}\OperatorTok{=}\NormalTok{J\_294[}\StringTok{\textquotesingle{}DV (V)\textquotesingle{}}\NormalTok{]}\OperatorTok{*}\DecValTok{10}\NormalTok{,}
\NormalTok{             fmt}\OperatorTok{=}\StringTok{\textquotesingle{}o\textquotesingle{}}\NormalTok{,label}\OperatorTok{=}\StringTok{\textquotesingle{}T = 295K\textquotesingle{}}\NormalTok{)}

\NormalTok{ax.errorbar(J\_ln2[}\StringTok{\textquotesingle{}R (ohms)\textquotesingle{}}\NormalTok{], J\_ln2[}\StringTok{\textquotesingle{}Vrms (V)\textquotesingle{}}\NormalTok{],}
\NormalTok{             yerr}\OperatorTok{=}\NormalTok{J\_ln2[}\StringTok{\textquotesingle{}DV (V)\textquotesingle{}}\NormalTok{] }\OperatorTok{*} \DecValTok{10}\NormalTok{,}
\NormalTok{             fmt}\OperatorTok{=}\StringTok{\textquotesingle{}o\textquotesingle{}}\NormalTok{, label}\OperatorTok{=}\StringTok{\textquotesingle{}T = 77K\textquotesingle{}}\NormalTok{)}

\NormalTok{ax.legend()}\OperatorTok{;} \CommentTok{\# Make a legend}
\end{Highlighting}
\end{Shaded}

\pandocbounded{\includegraphics[keepaspectratio]{Noise Analysis (template, 2024)_files/figure-latex/cell-8-output-1.png}}

\subsubsection{Include the other data}\label{include-the-other-data}

\begin{quote}
Repeat the lines in the cell above and include another data set so that
both the 395K and 77K data are on the same plot.
\end{quote}

\begin{Shaded}
\begin{Highlighting}[]
\CommentTok{\#\# This code was refactored to above.}

\CommentTok{\# \#\# Plot the data sets on one graph}
\CommentTok{\# \#\# Header commands provided}
\CommentTok{\#}
\CommentTok{\# \# Below shows how to make a plot with error bars.  The errors are multiplied by}
\CommentTok{\# \# 10 so that the bars are visible.}
\CommentTok{\#}
\CommentTok{\# \# If you have no error bars, simply omit the item \textquotesingle{}yerr=J\_294[\textquotesingle{}DV (V)\textquotesingle{}]*10\textquotesingle{}}
\CommentTok{\#}
\CommentTok{\# plt.errorbar(J\_ln2[\textquotesingle{}R (ohms)\textquotesingle{}], J\_ln2[\textquotesingle{}Vrms (V)\textquotesingle{}],}
\CommentTok{\#              yerr=J\_ln2[\textquotesingle{}DV (V)\textquotesingle{}] * 10,}
\CommentTok{\#              fmt=\textquotesingle{}o\textquotesingle{}, label=\textquotesingle{}T = 295K\textquotesingle{})}
\CommentTok{\# plt.legend();  \# Make a legend}
\end{Highlighting}
\end{Shaded}

\subsection{Exercise 2}\label{exercise-2}

\begin{quote}
\subsubsection{Part a. Modify the data}\label{part-a.-modify-the-data}
\end{quote}

\begin{quote}
Modify the data arrays to obtain the mean square voltages for each
temperature, and also the difference in the (squared) data for the two
temperatures, which will help remove the effects of noise in the
electronics. \textbf{Remember:} You have NumPy/Pandas arrays, so you can
do each task with a single line of code.
\end{quote}

\begin{quote}
Then plot the results, all on one plot so you can compare them visually.
\end{quote}

\begin{quote}
\mbox{}%
\paragraph{For data sets that have uncertainties associated with
them.}\label{for-data-sets-that-have-uncertainties-associated-with-them.}
\end{quote}

\begin{quote}
If you have uncertainties on each data point that you want to carry
forward in the analysis, when you square the value, the uncertainty is
NOT also squared. Instead it is multiplied by 2 times the
\textbar value\textbar. That is, if \(\sigma_x\) is the uncertainty in
\(x\), the uncertainty in \(x^2\) is \(\sigma_{x^2} = 2|x|\sigma_x\).
\end{quote}

\begin{quote}
Another way to work out the uncertainties is to first build arrays of
uncertainty objects from the data and uncertainty arrays. For example,
if the data are in an array called \texttt{X} and the uncertainty (i.e.,
error bars) are in an array called \texttt{sigma\_X}, you can build an
uncertainty array as follows:
\end{quote}

\begin{quote}
\# Import uNumPy functions. You could do this in the first cell import
uncertainties.unumpy as unp

\# Build an uncertainty array uX = unp.uarray(X, sigma\_X)

\# Square the array, and also propagate uncertainty uX\_sqrd = uX*uX

\# Access the parts of the uncertainty array. This is necessary for
curve fitting uX\_sqrd\_values = unp.nominal\_values(uX\_sqrd)
uX\_sqrd\_sigmas = unp.std\_devs(uX\_sqrd)
\end{quote}

\begin{Shaded}
\begin{Highlighting}[]
\CommentTok{\#\# Modify the arrays as specified above}
\CommentTok{\# Lets create the specified data:}
\ImportTok{import}\NormalTok{ uncertainties.unumpy }\ImportTok{as}\NormalTok{ unp}

\CommentTok{\# Lets start by viewing what is in the arrays}

\NormalTok{johnson\_294}
\end{Highlighting}
\end{Shaded}

\begin{longtable}[]{@{}llllllllll@{}}
\toprule\noalign{}
& 40k & 20k & 15k & 9.99k & 7.5k & 4.99k & 2.5k & 1.0k & 0.0K \\
\midrule\noalign{}
\endhead
\bottomrule\noalign{}
\endlastfoot
0 & 0.007909 & 0.005793 & 0.005050 & 0.004211 & 0.003701 & 0.002997 &
0.002196 & 0.001457 & 0.000620 \\
1 & 0.007962 & 0.005885 & 0.005047 & 0.004180 & 0.003633 & 0.003006 &
0.002207 & 0.001484 & 0.000624 \\
2 & 0.007796 & 0.005765 & 0.005031 & 0.004170 & 0.003611 & 0.002980 &
0.002148 & 0.001482 & 0.000615 \\
3 & 0.007801 & 0.005848 & 0.005163 & 0.004190 & 0.003652 & 0.003037 &
0.002192 & 0.001496 & 0.000629 \\
4 & 0.007821 & 0.005864 & 0.005107 & 0.004182 & 0.003747 & 0.002973 &
0.002194 & 0.001473 & 0.000625 \\
\end{longtable}

\begin{Shaded}
\begin{Highlighting}[]
\CommentTok{\# Let\textquotesingle{}s start by forming the uncertainty objects:}
\NormalTok{J\_294[}\StringTok{\textquotesingle{}Vrms (V) unc\textquotesingle{}}\NormalTok{] }\OperatorTok{=}\NormalTok{ unp.uarray(J\_294[}\StringTok{\textquotesingle{}Vrms (V)\textquotesingle{}}\NormalTok{], J\_294[}\StringTok{\textquotesingle{}DV (V)\textquotesingle{}}\NormalTok{])}
\NormalTok{J\_ln2[}\StringTok{\textquotesingle{}Vrms (V) unc\textquotesingle{}}\NormalTok{] }\OperatorTok{=}\NormalTok{ unp.uarray(J\_ln2[}\StringTok{\textquotesingle{}Vrms (V)\textquotesingle{}}\NormalTok{], J\_ln2[}\StringTok{\textquotesingle{}DV (V)\textquotesingle{}}\NormalTok{])}

\CommentTok{\# Now lets square the values}
\NormalTok{J\_294[}\StringTok{\textquotesingle{}Vms (V)\textquotesingle{}}\NormalTok{] }\OperatorTok{=}\NormalTok{ J\_294[}\StringTok{\textquotesingle{}Vrms (V) unc\textquotesingle{}}\NormalTok{] }\OperatorTok{**} \DecValTok{2}
\NormalTok{J\_ln2[}\StringTok{\textquotesingle{}Vms (V)\textquotesingle{}}\NormalTok{] }\OperatorTok{=}\NormalTok{ J\_ln2[}\StringTok{\textquotesingle{}Vrms (V) unc\textquotesingle{}}\NormalTok{] }\OperatorTok{**} \DecValTok{2}

\CommentTok{\# Finally lets take the difference of 295k and 77k}
\NormalTok{J\_diff }\OperatorTok{=}\NormalTok{ pd.DataFrame(\{}\StringTok{\textquotesingle{}R (ohms)\textquotesingle{}}\NormalTok{ : J\_294[}\StringTok{\textquotesingle{}R (ohms)\textquotesingle{}}\NormalTok{]\})}
\NormalTok{J\_diff[}\StringTok{\textquotesingle{}Vms (V) diff\textquotesingle{}}\NormalTok{] }\OperatorTok{=}\NormalTok{ J\_294[}\StringTok{\textquotesingle{}Vms (V)\textquotesingle{}}\NormalTok{] }\OperatorTok{{-}}\NormalTok{ J\_ln2[}\StringTok{\textquotesingle{}Vms (V)\textquotesingle{}}\NormalTok{]}

\NormalTok{J\_294}
\NormalTok{J\_diff}
\end{Highlighting}
\end{Shaded}

\begin{longtable}[]{@{}lll@{}}
\toprule\noalign{}
& R (ohms) & Vms (V) diff \\
\midrule\noalign{}
\endhead
\bottomrule\noalign{}
\endlastfoot
0 & 40000.0 & (4.31+/-0.12)e-05 \\
1 & 20000.0 & (2.40+/-0.06)e-05 \\
2 & 15000.0 & (1.83+/-0.06)e-05 \\
3 & 9990.0 & (1.243+/-0.019)e-05 \\
4 & 7500.0 & (9.5+/-0.4)e-06 \\
5 & 4990.0 & (5.91+/-0.17)e-06 \\
6 & 2500.0 & (3.20+/-0.11)e-06 \\
7 & 1000.0 & (1.31+/-0.05)e-06 \\
8 & 0.0 & (-1+/-8)e-09 \\
\end{longtable}

\begin{Shaded}
\begin{Highlighting}[]
\CommentTok{\#\# Plot the results}
\CommentTok{\#\# Header commands provided to format plot}
\CommentTok{\#\# Uncertainties here are also 10xed to be visible on the plot.}
\NormalTok{fig\_quad, ax\_quad }\OperatorTok{=}\NormalTok{ plt.subplots()}
\NormalTok{ax\_quad.grid(color}\OperatorTok{=}\StringTok{\textquotesingle{}\#2A3459\textquotesingle{}}\NormalTok{)}
\NormalTok{ax\_quad.set\_title(}\StringTok{\textquotesingle{}Johnson Noise Data, in Quadrature\textquotesingle{}}\NormalTok{)}
\NormalTok{ax\_quad.set\_ylabel(}\VerbatimStringTok{r\textquotesingle{}}\DecValTok{$}\VerbatimStringTok{V}\DecValTok{\^{}}\VerbatimStringTok{2\_J}\DecValTok{$}\VerbatimStringTok{ }\KeywordTok{(}\VerbatimStringTok{Vrms}\DecValTok{$\^{}}\VerbatimStringTok{2}\DecValTok{$}\KeywordTok{)}\VerbatimStringTok{\textquotesingle{}}\NormalTok{)}
\NormalTok{ax\_quad.set\_xlabel(}\VerbatimStringTok{r\textquotesingle{}Resistance }\DecValTok{$}\VerbatimStringTok{R}\DecValTok{$}\VerbatimStringTok{ }\KeywordTok{(}\DecValTok{$}\ErrorTok{\textbackslash{}}\VerbatimStringTok{Omega}\DecValTok{$}\KeywordTok{)}\VerbatimStringTok{\textquotesingle{}}\NormalTok{)}
\CommentTok{\#\# Add your code here}
\NormalTok{ax\_quad.errorbar(J\_294[}\StringTok{\textquotesingle{}R (ohms)\textquotesingle{}}\NormalTok{],}
\NormalTok{                 unp.nominal\_values(J\_294[}\StringTok{\textquotesingle{}Vms (V)\textquotesingle{}}\NormalTok{]),}
\NormalTok{                 unp.std\_devs(J\_294[}\StringTok{\textquotesingle{}Vms (V)\textquotesingle{}}\NormalTok{]) }\OperatorTok{*} \DecValTok{10}\NormalTok{,}
\NormalTok{                 label}\OperatorTok{=}\StringTok{\textquotesingle{}T = 295K\textquotesingle{}}\NormalTok{, fmt}\OperatorTok{=}\StringTok{\textquotesingle{}o\textquotesingle{}}\NormalTok{)}

\NormalTok{ax\_quad.errorbar(J\_ln2[}\StringTok{\textquotesingle{}R (ohms)\textquotesingle{}}\NormalTok{],}
\NormalTok{                 unp.nominal\_values(J\_ln2[}\StringTok{\textquotesingle{}Vms (V)\textquotesingle{}}\NormalTok{]),}
\NormalTok{                 unp.std\_devs(J\_ln2[}\StringTok{\textquotesingle{}Vms (V)\textquotesingle{}}\NormalTok{]) }\OperatorTok{*} \DecValTok{10}\NormalTok{,}
\NormalTok{                 label}\OperatorTok{=}\StringTok{\textquotesingle{}T = 77K\textquotesingle{}}\NormalTok{, fmt}\OperatorTok{=}\StringTok{\textquotesingle{}o\textquotesingle{}}\NormalTok{)}
\NormalTok{ax\_quad.errorbar(J\_diff[}\StringTok{\textquotesingle{}R (ohms)\textquotesingle{}}\NormalTok{],}
\NormalTok{                 unp.nominal\_values(J\_diff[}\StringTok{\textquotesingle{}Vms (V) diff\textquotesingle{}}\NormalTok{]),}
\NormalTok{                 unp.std\_devs(J\_diff[}\StringTok{\textquotesingle{}Vms (V) diff\textquotesingle{}}\NormalTok{]) }\OperatorTok{*} \DecValTok{10}\NormalTok{,}
\NormalTok{                 label}\OperatorTok{=}\StringTok{\textquotesingle{}T = diff\textquotesingle{}}\NormalTok{, fmt}\OperatorTok{=}\StringTok{\textquotesingle{}o\textquotesingle{}}\NormalTok{)}

\NormalTok{ax\_quad.legend()}\OperatorTok{;}
\end{Highlighting}
\end{Shaded}

\pandocbounded{\includegraphics[keepaspectratio]{Noise Analysis (template, 2024)_files/figure-latex/cell-13-output-1.png}}

\subsubsection{Part b. Fit the modified
data}\label{part-b.-fit-the-modified-data}

\begin{quote}
To fit the data set to a line, make use of the \textbf{LMFit} package.
It is a useful add-on to the SciPy fitting functions. This package
simplifies fitting data to a variety of standard functions. See the
\href{https://lmfit.github.io/lmfit-py/index.html}{Lmfit Documentation}
for a full discussion. The package is quite powerful, but for basic
fitting with common functions, it is very easy to use.
\end{quote}

\begin{quote}
\mbox{}%
\paragraph{Example: Fitting a line}\label{example-fitting-a-line}
\end{quote}

\begin{quote}
The example below shows how to use the package to fit data to a line,
obtain the fit parameters along with uncertainties, and then plot the
data and fit. Execute the cells and study how it works. (Note: the data
come from a calibration problem in physics 331)
\end{quote}

\begin{Shaded}
\begin{Highlighting}[]
\CommentTok{\# This cell only creates arrays of x and y data to feed to the fit example in the next cell.}
\CommentTok{\# Calibration Data from a physics 331 experiment.}
\CommentTok{\# First column is wavelength (nm), second is carriage position (cm)}
\CommentTok{\#}
\NormalTok{Cal\_data }\OperatorTok{=}\NormalTok{ np.array([}
\NormalTok{    [}\FloatTok{643.85}\NormalTok{, }\FloatTok{41.43}\NormalTok{],}
\NormalTok{    [}\FloatTok{579.07}\NormalTok{, }\FloatTok{37.24}\NormalTok{],}
\NormalTok{    [}\FloatTok{576.96}\NormalTok{, }\FloatTok{37.11}\NormalTok{],}
\NormalTok{    [}\FloatTok{546.08}\NormalTok{, }\FloatTok{35.10}\NormalTok{],}
\NormalTok{    [}\FloatTok{508.58}\NormalTok{, }\FloatTok{32.68}\NormalTok{],}
\NormalTok{    [}\FloatTok{479.99}\NormalTok{, }\FloatTok{30.83}\NormalTok{],}
\NormalTok{    [}\FloatTok{467.81}\NormalTok{, }\FloatTok{30.04}\NormalTok{],}
\NormalTok{    [}\FloatTok{435.83}\NormalTok{, }\FloatTok{27.96}\NormalTok{],}
\NormalTok{    [}\FloatTok{404.66}\NormalTok{, }\FloatTok{25.98}\NormalTok{]])}

\CommentTok{\# Array slicing separates x (position) and y (wavelength)}
\CommentTok{\# Goal of calibration is to be able to feed in a position and obtain a wavelength}
\NormalTok{wavelength }\OperatorTok{=}\NormalTok{ Cal\_data[:,}\DecValTok{0}\NormalTok{]}
\NormalTok{position }\OperatorTok{=}\NormalTok{ Cal\_data[:,}\DecValTok{1}\NormalTok{]}
\end{Highlighting}
\end{Shaded}

The following cell executes the fitting calculations.

\begin{Shaded}
\begin{Highlighting}[]
\CommentTok{\# imports a linear fitting model from lmfit  }
\CommentTok{\# ONLY IMPORT ONCE IN A NOTEBOOK}
\ImportTok{from}\NormalTok{ lmfit.models }\ImportTok{import}\NormalTok{ LinearModel, DimensionalError}

\CommentTok{\# create an instance of the model}
\CommentTok{\# You only need to do this once in a notebook}
\NormalTok{line }\OperatorTok{=}\NormalTok{ LinearModel()}

\CommentTok{\# One must have a guess of the parameters. The guess() method works with most of the standard}
\CommentTok{\# lmfit models}

\CommentTok{\# The return value is a Parameters structure.  See the documentation.}
\NormalTok{param\_guess }\OperatorTok{=}\NormalTok{ line.guess(wavelength, x}\OperatorTok{=}\NormalTok{position)}

\CommentTok{\# The line below executes the fitting process.  The results are returned to "line\_fit"}
\NormalTok{line\_fit }\OperatorTok{=}\NormalTok{ line.fit(wavelength, param\_guess, x}\OperatorTok{=}\NormalTok{position)}

\CommentTok{\# This prints the results in an easy to read form}
\BuiltInTok{print}\NormalTok{(line\_fit.fit\_report())}

\CommentTok{\#Then you can plot the results quickly just to see how it looks using the plot() method}
\NormalTok{line\_fit.plot()}
\CommentTok{\# Optional: Change axis labels from default \textquotesingle{}X\textquotesingle{} vs. \textquotesingle{}Y\textquotesingle{}.}
\NormalTok{plt.xlabel(}\StringTok{\textquotesingle{}Carriage position (cm)\textquotesingle{}}\NormalTok{)}
\NormalTok{plt.ylabel(}\StringTok{\textquotesingle{}Emission wavelength (nm)\textquotesingle{}}\NormalTok{)}\OperatorTok{;}
\end{Highlighting}
\end{Shaded}

\begin{verbatim}
[[Model]]
    Model(linear)
[[Fit Statistics]]
    # fitting method   = leastsq
    # function evals   = 4
    # data points      = 9
    # variables        = 2
    chi-square         = 0.18041902
    reduced chi-square = 0.02577415
    Akaike info crit   = -31.1872805
    Bayesian info crit = -30.7928313
    R-squared          = 0.99999620
[[Variables]]
    slope:      15.4639413 +/- 0.01139807 (0.07%) (init = 15.46394)
    intercept:  3.20598011 +/- 0.38164182 (11.90%) (init = 3.20598)
[[Correlations]] (unreported correlations are < 0.100)
    C(slope, intercept) = -0.9901
\end{verbatim}

\pandocbounded{\includegraphics[keepaspectratio]{Noise Analysis (template, 2024)_files/figure-latex/cell-15-output-2.png}}

Fit each to a line and obtain the slope with uncertainty. Plot the data
with the fit lines.

First, I'll make functions to clean up the coding.

\begin{Shaded}
\begin{Highlighting}[]
\CommentTok{\#\# Defines a function to do the work.  Study it.  If you don\textquotesingle{}t understand how this works,}
\CommentTok{\#\# find out by asking questions and or studying the functions in the code.}

\KeywordTok{def}\NormalTok{ line\_fit\_and\_plot(xdata, ydata, yerr}\OperatorTok{=}\VariableTok{None}\NormalTok{, model}\OperatorTok{=}\NormalTok{LinearModel(), xlabel}\OperatorTok{=}\StringTok{\textquotesingle{}X\textquotesingle{}}\NormalTok{, ylabel}\OperatorTok{=}\StringTok{\textquotesingle{}Y\textquotesingle{}}\NormalTok{):}
    \CommentTok{\textquotesingle{}\textquotesingle{}\textquotesingle{}}
\CommentTok{    Fit a line or curve, and plot/show the fit results.}
\CommentTok{    The function returns a parameters object with the fit parameters}
\CommentTok{    \textquotesingle{}\textquotesingle{}\textquotesingle{}}
\NormalTok{    param\_guess }\OperatorTok{=}\NormalTok{ model.guess(ydata, x}\OperatorTok{=}\NormalTok{xdata)}
    \ControlFlowTok{if}\NormalTok{ (yerr }\KeywordTok{is} \VariableTok{None}\NormalTok{):}
\NormalTok{        model\_fit }\OperatorTok{=}\NormalTok{ model.fit(ydata, param\_guess, x}\OperatorTok{=}\NormalTok{xdata)}
    \ControlFlowTok{else}\NormalTok{:}
\NormalTok{        model\_fit }\OperatorTok{=}\NormalTok{ model.fit(ydata, param\_guess, x}\OperatorTok{=}\NormalTok{xdata, weights}\OperatorTok{=}\DecValTok{1}\OperatorTok{/}\NormalTok{yerr)}
    \BuiltInTok{print}\NormalTok{(model\_fit.fit\_report(show\_correl}\OperatorTok{=}\VariableTok{False}\NormalTok{))}
\NormalTok{    model\_fit.plot()}
\NormalTok{    plt.xlabel(xlabel)}
\NormalTok{    plt.ylabel(ylabel)}\OperatorTok{;}
    \ControlFlowTok{return}\NormalTok{ model\_fit.params}

\CommentTok{\#\# This function use the Uncertainties function to make an uncertainty object}

\KeywordTok{def}\NormalTok{ get\_uslope(params):}
    \ControlFlowTok{return}\NormalTok{ unc.ufloat(params[}\StringTok{\textquotesingle{}slope\textquotesingle{}}\NormalTok{].value, params[}\StringTok{\textquotesingle{}slope\textquotesingle{}}\NormalTok{].stderr)}
\end{Highlighting}
\end{Shaded}

Then run the functions.

\subsubsection{First, for the 300k Data:}\label{first-for-the-300k-data}

\begin{Shaded}
\begin{Highlighting}[]
\CommentTok{\#\# Use the functions above to run the fit for the modified 295K data}
\CommentTok{\#\# and save the fit parameters.  Then pull out the slope}
\CommentTok{\#\# Here is how you would use the above functions with the example data:}

\CommentTok{\# Run the fit}
\CommentTok{\# example\_fit\_params = line\_fit\_and\_plot(position,wavelength,}
\CommentTok{\#                                        xlabel=\textquotesingle{}Carriage position (cm)\textquotesingle{},ylabel=\textquotesingle{}Emission wavelength (nm)\textquotesingle{})}

\NormalTok{J\_294\_fit\_params }\OperatorTok{=}\NormalTok{ line\_fit\_and\_plot(J\_294[}\StringTok{\textquotesingle{}R (ohms)\textquotesingle{}}\NormalTok{],}
\NormalTok{                                     unp.nominal\_values(J\_294[}\StringTok{\textquotesingle{}Vms (V)\textquotesingle{}}\NormalTok{]),}
\NormalTok{                                     unp.std\_devs(J\_294[}\StringTok{\textquotesingle{}Vms (V)\textquotesingle{}}\NormalTok{]),)}


\CommentTok{\# Obtain the slope and its uncertainty into an uncertainty object}
\CommentTok{\# slope\_with\_uncertainty = get\_uslope(example\_fit\_params)}
\NormalTok{slope\_with\_uncertainty }\OperatorTok{=}\NormalTok{ get\_uslope(J\_294\_fit\_params)}
\BuiltInTok{print}\NormalTok{(}\StringTok{\textquotesingle{}}\CharTok{\textbackslash{}n}\StringTok{Slope = \{:.2uP\}\textquotesingle{}}\NormalTok{.}\BuiltInTok{format}\NormalTok{(slope\_with\_uncertainty))}
\end{Highlighting}
\end{Shaded}

\begin{verbatim}
[[Model]]
    Model(linear)
[[Fit Statistics]]
    # fitting method   = leastsq
    # function evals   = 7
    # data points      = 9
    # variables        = 2
    chi-square         = 42.9965765
    reduced chi-square = 6.14236807
    Akaike info crit   = 18.0750633
    Bayesian info crit = 18.4695124
    R-squared          = 0.98552059
[[Variables]]
    slope:      1.6997e-09 +/- 2.3043e-11 (1.36%) (init = 1.544299e-09)
    intercept:  3.9117e-07 +/- 1.5978e-08 (4.08%) (init = 1.438028e-06)

Slope = (1.700±0.023)×10⁻⁹
\end{verbatim}

\pandocbounded{\includegraphics[keepaspectratio]{Noise Analysis (template, 2024)_files/figure-latex/cell-17-output-2.png}}

\subsubsection{For the 77k Data}\label{for-the-77k-data}

\begin{Shaded}
\begin{Highlighting}[]
\NormalTok{J\_ln2\_fit\_params }\OperatorTok{=}\NormalTok{ line\_fit\_and\_plot(J\_ln2[}\StringTok{\textquotesingle{}R (ohms)\textquotesingle{}}\NormalTok{],}
\NormalTok{                                     unp.nominal\_values(J\_ln2[}\StringTok{\textquotesingle{}Vms (V)\textquotesingle{}}\NormalTok{]),}
\NormalTok{                                     unp.std\_devs(J\_ln2[}\StringTok{\textquotesingle{}Vms (V)\textquotesingle{}}\NormalTok{]),)}


\CommentTok{\# Obtain the slope and its uncertainty into an uncertainty object}
\CommentTok{\# slope\_with\_uncertainty = get\_uslope(example\_fit\_params)}
\NormalTok{slope\_with\_uncertainty }\OperatorTok{=}\NormalTok{ get\_uslope(J\_ln2\_fit\_params)}
\BuiltInTok{print}\NormalTok{(}\StringTok{\textquotesingle{}}\CharTok{\textbackslash{}n}\StringTok{Slope = \{:.2uP\}\textquotesingle{}}\NormalTok{.}\BuiltInTok{format}\NormalTok{(slope\_with\_uncertainty))}
\end{Highlighting}
\end{Shaded}

\begin{verbatim}
[[Model]]
    Model(linear)
[[Fit Statistics]]
    # fitting method   = leastsq
    # function evals   = 7
    # data points      = 9
    # variables        = 2
    chi-square         = 19.9694049
    reduced chi-square = 2.85277212
    Akaike info crit   = 11.1727909
    Bayesian info crit = 11.5672401
    R-squared          = 0.99715304
[[Variables]]
    slope:      4.7755e-10 +/- 7.1432e-12 (1.50%) (init = 4.580018e-10)
    intercept:  3.9135e-07 +/- 8.4604e-09 (2.16%) (init = 5.496959e-07)

Slope = (4.776±0.071)×10⁻¹⁰
\end{verbatim}

\pandocbounded{\includegraphics[keepaspectratio]{Noise Analysis (template, 2024)_files/figure-latex/cell-18-output-2.png}}

\subsubsection{For the Diff Data}\label{for-the-diff-data}

\begin{Shaded}
\begin{Highlighting}[]

\CommentTok{\# Finally for the diff dataset"}

\NormalTok{J\_diff\_fit\_params }\OperatorTok{=}\NormalTok{ line\_fit\_and\_plot(J\_diff[}\StringTok{\textquotesingle{}R (ohms)\textquotesingle{}}\NormalTok{],}
\NormalTok{                                     unp.nominal\_values(J\_diff[}\StringTok{\textquotesingle{}Vms (V) diff\textquotesingle{}}\NormalTok{]),}
\NormalTok{                                     unp.std\_devs(J\_diff[}\StringTok{\textquotesingle{}Vms (V) diff\textquotesingle{}}\NormalTok{]),)}


\CommentTok{\# Obtain the slope and its uncertainty into an uncertainty object}
\CommentTok{\# slope\_with\_uncertainty = get\_uslope(example\_fit\_params)}
\NormalTok{slope\_with\_uncertainty }\OperatorTok{=}\NormalTok{ get\_uslope(J\_diff\_fit\_params)}
\BuiltInTok{print}\NormalTok{(}\StringTok{\textquotesingle{}}\CharTok{\textbackslash{}n}\StringTok{Slope = \{:.2uP\}\textquotesingle{}}\NormalTok{.}\BuiltInTok{format}\NormalTok{(slope\_with\_uncertainty))}
\end{Highlighting}
\end{Shaded}

\begin{verbatim}
[[Model]]
    Model(linear)
[[Fit Statistics]]
    # fitting method   = leastsq
    # function evals   = 7
    # data points      = 9
    # variables        = 2
    chi-square         = 30.0805152
    reduced chi-square = 4.29721646
    Akaike info crit   = 14.8598775
    Bayesian info crit = 15.2543266
    R-squared          = 0.97983080
[[Variables]]
    slope:      1.2139e-09 +/- 2.3781e-11 (1.96%) (init = 1.086297e-09)
    intercept:  2.6615e-09 +/- 1.7086e-08 (641.99%) (init = 8.883324e-07)

Slope = (1.214±0.024)×10⁻⁹
\end{verbatim}

\pandocbounded{\includegraphics[keepaspectratio]{Noise Analysis (template, 2024)_files/figure-latex/cell-19-output-2.png}}

\subsubsection{Calculate a Boltzmann
constant}\label{calculate-a-boltzmann-constant}

From the results, calculate the implied Boltzmann constant (with
uncertainty).

Revised gain of low-noise amplifier \(G=10122\pm35\) (as of July 2021,
DBP)

\begin{Shaded}
\begin{Highlighting}[]
\CommentTok{\#\# Create uncertainties objects for the other quantities.  The first two are examples}
\NormalTok{T\_295 }\OperatorTok{=}\NormalTok{ unc.ufloat(}\FloatTok{295.0}\NormalTok{,}\FloatTok{1.0}\NormalTok{) }\CommentTok{\# K}
\NormalTok{G }\OperatorTok{=}\NormalTok{ unc.ufloat(}\DecValTok{10122}\NormalTok{,}\DecValTok{35}\NormalTok{) }\CommentTok{\# unitless}
\NormalTok{k\_B }\OperatorTok{=}\NormalTok{ const.Boltzmann }\CommentTok{\# J/K Accepted value of Boltsmann constant from SciPy constants library.}
\CommentTok{\# You do the rest}

\NormalTok{k\_295 }\OperatorTok{=}\NormalTok{ get\_uslope(J\_294\_fit\_params) }\OperatorTok{/}\NormalTok{ (G}\OperatorTok{**}\DecValTok{2} \OperatorTok{*} \DecValTok{4} \OperatorTok{*}\NormalTok{ T\_295 }\OperatorTok{*} \DecValTok{1000}\NormalTok{)}

\CommentTok{\#\# Calculate and print k\_Boltzmann}
\CommentTok{\# Use the following print line:}
\BuiltInTok{print}\NormalTok{(}\StringTok{\textquotesingle{}Boltzmann constant from T = 295K data = \{:.2uP\} J/K\textquotesingle{}}\NormalTok{.}\BuiltInTok{format}\NormalTok{(k\_295))}
\BuiltInTok{print}\NormalTok{(}\StringTok{\textquotesingle{}Accepted value = }\SpecialCharTok{\{:.4g\}}\StringTok{ J/K\textquotesingle{}}\NormalTok{.}\BuiltInTok{format}\NormalTok{(k\_B))}

\DecValTok{1} \OperatorTok{/}\NormalTok{ (}\DecValTok{4} \OperatorTok{*}\NormalTok{ T\_295 }\OperatorTok{*} \DecValTok{1000}\NormalTok{)}
\end{Highlighting}
\end{Shaded}

\begin{verbatim}
Boltzmann constant from T = 295K data = (1.406±0.022)×10⁻²³ J/K
Accepted value = 1.381e-23 J/K
\end{verbatim}

\begin{verbatim}
8.474576271186441e-07+/-2.872737719046251e-09
\end{verbatim}

\subsubsection{77 K data}\label{k-data}

Repeat the process for the 77K data set.

\begin{Shaded}
\begin{Highlighting}[]
\CommentTok{\#\# Repeat for the 77K data}

\NormalTok{T\_77 }\OperatorTok{=}\NormalTok{ unc.ufloat(}\FloatTok{77.0}\NormalTok{,}\FloatTok{1.0}\NormalTok{)}
\NormalTok{k\_ln2 }\OperatorTok{=}\NormalTok{ get\_uslope(J\_ln2\_fit\_params) }\OperatorTok{/}\NormalTok{ (G}\OperatorTok{**}\DecValTok{2} \OperatorTok{*} \DecValTok{4} \OperatorTok{*}\NormalTok{ T\_77 }\OperatorTok{*} \DecValTok{1000}\NormalTok{)}

\BuiltInTok{print}\NormalTok{(}\StringTok{\textquotesingle{}Boltzmann constant from T = 77K data = \{:.2uP\} J/K\textquotesingle{}}\NormalTok{.}\BuiltInTok{format}\NormalTok{(k\_ln2))}
\BuiltInTok{print}\NormalTok{(}\StringTok{\textquotesingle{}Accepted value = }\SpecialCharTok{\{:.4g\}}\StringTok{ J/K\textquotesingle{}}\NormalTok{.}\BuiltInTok{format}\NormalTok{(k\_B))}
\end{Highlighting}
\end{Shaded}

\begin{verbatim}
Boltzmann constant from T = 77K data = (1.513±0.032)×10⁻²³ J/K
Accepted value = 1.381e-23 J/K
\end{verbatim}

And finally, the difference data

\begin{Shaded}
\begin{Highlighting}[]
\CommentTok{\#\# Repeat for the "difference" data (295K{-}77K) subracted in quadrature}
\NormalTok{T\_diff }\OperatorTok{=}\NormalTok{ T\_295 }\OperatorTok{{-}}\NormalTok{ T\_77}
\NormalTok{k\_diff }\OperatorTok{=}\NormalTok{ get\_uslope(J\_diff\_fit\_params) }\OperatorTok{/}\NormalTok{ (G}\OperatorTok{**}\DecValTok{2} \OperatorTok{*} \DecValTok{4} \OperatorTok{*}\NormalTok{ T\_diff }\OperatorTok{*} \DecValTok{1000}\NormalTok{)}

\BuiltInTok{print}\NormalTok{(}\StringTok{\textquotesingle{}Boltzmann constant from T = 77K data = \{:.2uP\} J/K\textquotesingle{}}\NormalTok{.}\BuiltInTok{format}\NormalTok{(k\_diff))}
\BuiltInTok{print}\NormalTok{(}\StringTok{\textquotesingle{}Accepted value = }\SpecialCharTok{\{:.4g\}}\StringTok{ J/K\textquotesingle{}}\NormalTok{.}\BuiltInTok{format}\NormalTok{(k\_B))}
\end{Highlighting}
\end{Shaded}

\begin{verbatim}
Boltzmann constant from T = 77K data = (1.359±0.030)×10⁻²³ J/K
Accepted value = 1.381e-23 J/K
\end{verbatim}

\subsubsection{Plot everything on one
graph}\label{plot-everything-on-one-graph}

Make a single plot that shows all three sets of data (as points) and the
three fit lines (as lines). Include a legend.

The cell below shows how to create a fit line using an arbitrary set of
x-values based on the range of x data. It uses the example data sets.

\paragraph{Recall}\label{recall}

Let's start with the plot \texttt{ax\_grid}, which, if we recall, is the
graph with all the datapoints. we have 3 sets of line data that we will
now fit into these graphs: * \_295 300k * \_ln2 77k * \_diff difference
in the data.

Lets view our original graph first:

\begin{Shaded}
\begin{Highlighting}[]
\CommentTok{\#\# Lets replot based on the Gain.}
\CommentTok{\# fig\_john, ax\_john = plt.subplots()}
\CommentTok{\#}
\CommentTok{\#}
\CommentTok{\# ax\_john.grid()}
\CommentTok{\# ax\_john.set\_title(\textquotesingle{}Johnson Noise Data, in Quadrature\textquotesingle{})}
\CommentTok{\# ax\_john.set\_ylabel(r\textquotesingle{}$V\^{}2\_J$ (Vrms$\^{}2$)\textquotesingle{})}
\CommentTok{\# ax\_john.set\_xlabel(r\textquotesingle{}Resistance $R$ ($\textbackslash{}Omega$)\textquotesingle{})}
\CommentTok{\# ax\_john.set\_yli}
\CommentTok{\# ax\_john.autoscale(enable=True, axis=ax\_john.get\_yaxis())}
\CommentTok{\# \#}
\CommentTok{\# ax\_john.errorbar(J\_294[\textquotesingle{}R (ohms)\textquotesingle{}],}
\CommentTok{\#                  unp.nominal\_values(J\_294[\textquotesingle{}Vms (V)\textquotesingle{}] / G),}
\CommentTok{\#                  unp.std\_devs(J\_294[\textquotesingle{}Vms (V)\textquotesingle{}]) * 10,}
\CommentTok{\#                  label=\textquotesingle{}T = 295K\textquotesingle{}, fmt=\textquotesingle{}o\textquotesingle{})}

\CommentTok{\# ax\_john.errorbar(J\_ln2[\textquotesingle{}R (ohms)\textquotesingle{}],}
\CommentTok{\#                  unp.nominal\_values(J\_ln2[\textquotesingle{}Vms (V)\textquotesingle{}]),}
\CommentTok{\#                  unp.std\_devs(J\_ln2[\textquotesingle{}Vms (V)\textquotesingle{}]) * 10,}
\CommentTok{\#                  label=\textquotesingle{}T = 77K\textquotesingle{}, fmt=\textquotesingle{}o\textquotesingle{})}
\CommentTok{\# ax\_john.errorbar(J\_diff[\textquotesingle{}R (ohms)\textquotesingle{}],}
\CommentTok{\#                  unp.nominal\_values(J\_diff[\textquotesingle{}Vms (V) diff\textquotesingle{}]),}
\CommentTok{\#                  unp.std\_devs(J\_diff[\textquotesingle{}Vms (V) diff\textquotesingle{}]) * 10,}
\CommentTok{\#                  label=\textquotesingle{}T = diff\textquotesingle{}, fmt=\textquotesingle{}o\textquotesingle{})}

\NormalTok{plt.show()}
\end{Highlighting}
\end{Shaded}

\paragraph{Adding Fit lines}\label{adding-fit-lines}

\begin{Shaded}
\begin{Highlighting}[]
\BuiltInTok{type}\NormalTok{(J\_294[}\StringTok{\textquotesingle{}R (ohms)\textquotesingle{}}\NormalTok{].}\BuiltInTok{min}\NormalTok{())}
\end{Highlighting}
\end{Shaded}

\begin{verbatim}
numpy.float64
\end{verbatim}

\begin{Shaded}
\begin{Highlighting}[]
\CommentTok{\# start with \_294}

\NormalTok{xvalues\_294 }\OperatorTok{=}\NormalTok{ np.linspace(J\_294[}\StringTok{\textquotesingle{}R (ohms)\textquotesingle{}}\NormalTok{].}\BuiltInTok{min}\NormalTok{(), J\_294[}\StringTok{\textquotesingle{}R (ohms)\textquotesingle{}}\NormalTok{].}\BuiltInTok{max}\NormalTok{(), }\DecValTok{100}\NormalTok{)}
\NormalTok{yvalues\_294 }\OperatorTok{=}\NormalTok{ line.}\BuiltInTok{eval}\NormalTok{(J\_294\_fit\_params, x}\OperatorTok{=}\NormalTok{xvalues\_294)}


\NormalTok{xvalues\_ln2 }\OperatorTok{=}\NormalTok{ np.linspace(J\_ln2[}\StringTok{\textquotesingle{}R (ohms)\textquotesingle{}}\NormalTok{].}\BuiltInTok{min}\NormalTok{(), J\_ln2[}\StringTok{\textquotesingle{}R (ohms)\textquotesingle{}}\NormalTok{].}\BuiltInTok{max}\NormalTok{(), }\DecValTok{100}\NormalTok{)}
\NormalTok{yvalues\_ln2 }\OperatorTok{=}\NormalTok{ line.}\BuiltInTok{eval}\NormalTok{(J\_ln2\_fit\_params, x}\OperatorTok{=}\NormalTok{xvalues\_ln2)}


\NormalTok{xvalues\_diff }\OperatorTok{=}\NormalTok{ np.linspace(J\_diff[}\StringTok{\textquotesingle{}R (ohms)\textquotesingle{}}\NormalTok{].}\BuiltInTok{min}\NormalTok{(), J\_diff[}\StringTok{\textquotesingle{}R (ohms)\textquotesingle{}}\NormalTok{].}\BuiltInTok{max}\NormalTok{(), }\DecValTok{100}\NormalTok{)}
\NormalTok{yvalues\_diff }\OperatorTok{=}\NormalTok{ line.}\BuiltInTok{eval}\NormalTok{(J\_diff\_fit\_params, x}\OperatorTok{=}\NormalTok{xvalues\_diff)}


\NormalTok{yvalues\_294}
\end{Highlighting}
\end{Shaded}

\begin{verbatim}
array([3.91171144e-07, 1.07791840e-06, 1.76466565e-06, 2.45141290e-06,
       3.13816015e-06, 3.82490740e-06, 4.51165466e-06, 5.19840191e-06,
       5.88514916e-06, 6.57189641e-06, 7.25864366e-06, 7.94539091e-06,
       8.63213817e-06, 9.31888542e-06, 1.00056327e-05, 1.06923799e-05,
       1.13791272e-05, 1.20658744e-05, 1.27526217e-05, 1.34393689e-05,
       1.41261162e-05, 1.48128634e-05, 1.54996107e-05, 1.61863579e-05,
       1.68731052e-05, 1.75598524e-05, 1.82465997e-05, 1.89333469e-05,
       1.96200942e-05, 2.03068414e-05, 2.09935887e-05, 2.16803360e-05,
       2.23670832e-05, 2.30538305e-05, 2.37405777e-05, 2.44273250e-05,
       2.51140722e-05, 2.58008195e-05, 2.64875667e-05, 2.71743140e-05,
       2.78610612e-05, 2.85478085e-05, 2.92345557e-05, 2.99213030e-05,
       3.06080502e-05, 3.12947975e-05, 3.19815447e-05, 3.26682920e-05,
       3.33550392e-05, 3.40417865e-05, 3.47285337e-05, 3.54152810e-05,
       3.61020282e-05, 3.67887755e-05, 3.74755227e-05, 3.81622700e-05,
       3.88490173e-05, 3.95357645e-05, 4.02225118e-05, 4.09092590e-05,
       4.15960063e-05, 4.22827535e-05, 4.29695008e-05, 4.36562480e-05,
       4.43429953e-05, 4.50297425e-05, 4.57164898e-05, 4.64032370e-05,
       4.70899843e-05, 4.77767315e-05, 4.84634788e-05, 4.91502260e-05,
       4.98369733e-05, 5.05237205e-05, 5.12104678e-05, 5.18972150e-05,
       5.25839623e-05, 5.32707095e-05, 5.39574568e-05, 5.46442040e-05,
       5.53309513e-05, 5.60176985e-05, 5.67044458e-05, 5.73911931e-05,
       5.80779403e-05, 5.87646876e-05, 5.94514348e-05, 6.01381821e-05,
       6.08249293e-05, 6.15116766e-05, 6.21984238e-05, 6.28851711e-05,
       6.35719183e-05, 6.42586656e-05, 6.49454128e-05, 6.56321601e-05,
       6.63189073e-05, 6.70056546e-05, 6.76924018e-05, 6.83791491e-05])
\end{verbatim}

\begin{Shaded}
\begin{Highlighting}[]
\CommentTok{\# Finally, we plot:}

\NormalTok{ax\_quad.plot(xvalues\_294, yvalues\_294, color}\OperatorTok{=}\StringTok{\textquotesingle{}blue\textquotesingle{}}\NormalTok{, label}\OperatorTok{=}\StringTok{\textquotesingle{}295K fit\textquotesingle{}}\NormalTok{)}
\NormalTok{ax\_quad.plot(xvalues\_ln2, yvalues\_ln2, color}\OperatorTok{=}\StringTok{\textquotesingle{}yellow\textquotesingle{}}\NormalTok{, label}\OperatorTok{=}\StringTok{\textquotesingle{}77K fit\textquotesingle{}}\NormalTok{)}
\NormalTok{ax\_quad.plot(xvalues\_diff, yvalues\_diff, color}\OperatorTok{=}\StringTok{\textquotesingle{}purple\textquotesingle{}}\NormalTok{, label}\OperatorTok{=}\StringTok{\textquotesingle{}diff fit\textquotesingle{}}\NormalTok{)}
\NormalTok{ax\_quad.legend()}
\NormalTok{ax\_quad.grid(color}\OperatorTok{=}\StringTok{\textquotesingle{}\#2A3459\textquotesingle{}}\NormalTok{)}
\CommentTok{\# Let\textquotesingle{}s also add the slope data:}



\NormalTok{display(fig\_quad)}
\end{Highlighting}
\end{Shaded}

\pandocbounded{\includegraphics[keepaspectratio]{Noise Analysis (template, 2024)_files/figure-latex/cell-26-output-1.png}}

\subsubsection{Part c.}\label{part-c.}

Summary of results for Boltsmann constant:

\begin{Shaded}
\begin{Highlighting}[]
\CommentTok{\#\# Summarize the results in one table}
\CommentTok{\#\# Like so:}
\BuiltInTok{print}\NormalTok{(}\StringTok{\textquotesingle{}  T (K)  |  k\_B (J/K)   \textquotesingle{}}\NormalTok{)}
\BuiltInTok{print}\NormalTok{(}\StringTok{\textquotesingle{}{-}{-}{-}{-}{-}{-}{-}{-}{-}|{-}{-}{-}{-}{-}{-}{-}{-}{-}{-}{-}{-}{-}{-}{-}{-}{-}{-}{-}{-}\textquotesingle{}}\NormalTok{)}
\BuiltInTok{print}\NormalTok{(}\StringTok{\textquotesingle{}   295   | \{:.1uP\}\textquotesingle{}}\NormalTok{.}\BuiltInTok{format}\NormalTok{(k\_295))}
\BuiltInTok{print}\NormalTok{(}\StringTok{\textquotesingle{}    77   | \{:.1uP\}\textquotesingle{}}\NormalTok{.}\BuiltInTok{format}\NormalTok{(k\_ln2))}
\BuiltInTok{print}\NormalTok{(}\StringTok{\textquotesingle{} 295{-}77  | \{:.1uP\}\textquotesingle{}}\NormalTok{.}\BuiltInTok{format}\NormalTok{(k\_diff))}
\BuiltInTok{print}\NormalTok{(}\StringTok{\textquotesingle{}Accepted | }\SpecialCharTok{\{:10.4g\}}\StringTok{\textquotesingle{}}\NormalTok{.}\BuiltInTok{format}\NormalTok{(k\_B))}
\end{Highlighting}
\end{Shaded}

\begin{verbatim}
  T (K)  |  k_B (J/K)   
---------|--------------------
   295   | (1.41±0.02)×10⁻²³
    77   | (1.51±0.03)×10⁻²³
 295-77  | (1.36±0.03)×10⁻²³
Accepted |  1.381e-23
\end{verbatim}

\subsection{Exercise 3: Noise Figure}\label{exercise-3-noise-figure}

Calculate the ``noise figure'' for the low-noise amp, as described in
the instructions.

The noise figure is defined:

\[ NF = 20\log_{10}\frac{V_{rms}(R)}{G\times\sqrt{4k_BTRB}} \; \text{dB}\]

Please limit the noise figure to 2 digits beyond the decimal point.

Note: It clearly does not work for \(R=0\). You will need to leave this
out of the calculations.

\begin{Shaded}
\begin{Highlighting}[]

\ImportTok{from}\NormalTok{ uncertainties }\ImportTok{import}\NormalTok{ umath}
\CommentTok{\#\# Calculate the Noise figure for the various values of R at}
\CommentTok{\#\# room temperature and display it as a table or a plot}

\CommentTok{\#\# Lets define a function to handle everything for us:}
\end{Highlighting}
\end{Shaded}

\begin{Shaded}
\begin{Highlighting}[]

\KeywordTok{def}\NormalTok{ NF\_calc(R, Vrms, T, gain }\OperatorTok{=}\NormalTok{ G, k\_boltz }\OperatorTok{=}\NormalTok{ k\_B):}
    \ControlFlowTok{if}\NormalTok{ (R }\OperatorTok{==} \DecValTok{0}\NormalTok{):}
        \ControlFlowTok{return}\NormalTok{ np.nan}\CommentTok{\# For the 0 Edge Case}
    \ControlFlowTok{return}\NormalTok{  Vrms }\OperatorTok{/}\NormalTok{ (gain }\OperatorTok{*}\NormalTok{ umath.sqrt(}\DecValTok{4} \OperatorTok{*}\NormalTok{ k\_boltz }\OperatorTok{*}\NormalTok{ T }\OperatorTok{*}\NormalTok{ R }\OperatorTok{*} \DecValTok{1000}\NormalTok{ ))}

\KeywordTok{def}\NormalTok{ nf\_izer(target0, target1, temp):}
\NormalTok{    result }\OperatorTok{=} \BuiltInTok{list}\NormalTok{()}
    \ControlFlowTok{if} \BuiltInTok{len}\NormalTok{(target0) }\OperatorTok{!=} \BuiltInTok{len}\NormalTok{(target1):}
        \ControlFlowTok{raise} \PreprocessorTok{ValueError}\NormalTok{(}\StringTok{"Not the right length"}\NormalTok{)}
    \ControlFlowTok{for}\NormalTok{ i }\KeywordTok{in} \BuiltInTok{range}\NormalTok{(}\BuiltInTok{len}\NormalTok{(target0)):}
\NormalTok{        result.append(NF\_calc(target0.iloc[i], target1.iloc[i], temp))}
    \ControlFlowTok{return}\NormalTok{ result}

\NormalTok{NF\_294 }\OperatorTok{=}\NormalTok{ nf\_izer(J\_294[}\StringTok{\textquotesingle{}R (ohms)\textquotesingle{}}\NormalTok{], J\_294[}\StringTok{\textquotesingle{}Vrms (V)\textquotesingle{}}\NormalTok{], T\_295)}
\NormalTok{NF\_ln2 }\OperatorTok{=}\NormalTok{ nf\_izer(J\_ln2[}\StringTok{\textquotesingle{}R (ohms)\textquotesingle{}}\NormalTok{], J\_ln2[}\StringTok{\textquotesingle{}Vrms (V)\textquotesingle{}}\NormalTok{], T\_77)}
\NormalTok{NF\_diff }\OperatorTok{=}\NormalTok{ nf\_izer(J\_diff[}\StringTok{\textquotesingle{}R (ohms)\textquotesingle{}}\NormalTok{], J\_diff[}\StringTok{\textquotesingle{}Vms (V) diff\textquotesingle{}}\NormalTok{].}\BuiltInTok{apply}\NormalTok{(np.}\BuiltInTok{abs}\NormalTok{).}\BuiltInTok{apply}\NormalTok{(umath.sqrt), T\_diff)}

\NormalTok{summary }\OperatorTok{=}\NormalTok{ pd.DataFrame(\{}\StringTok{\textquotesingle{}R (ohms)\textquotesingle{}}\NormalTok{: J\_294[}\StringTok{\textquotesingle{}R (ohms)\textquotesingle{}}\NormalTok{], }\StringTok{\textquotesingle{}nf\_294\textquotesingle{}}\NormalTok{ : NF\_294, }\StringTok{\textquotesingle{}nf\_ln2\textquotesingle{}}\NormalTok{ : NF\_ln2, }\StringTok{\textquotesingle{}nf\_diff\textquotesingle{}}\NormalTok{ : NF\_diff\})}

\NormalTok{summary}
\CommentTok{\# first, lets create staging arrays}
\CommentTok{\# NF\_294 = np.zeros\_like(J\_294[\textquotesingle{}R (ohms)\textquotesingle{}])}
\CommentTok{\# NF\_ln2 = np.zeros\_like(J\_ln2[\textquotesingle{}R (ohms)\textquotesingle{}])}
\CommentTok{\# NF\_diff = np.zeros\_like(J\_diff[\textquotesingle{}R (ohms)\textquotesingle{}])}

\CommentTok{\# now,}


\CommentTok{\# Next we\textquotesingle{}ll create a smplified dataframe}
\CommentTok{\# big = pd.DataFrame(\{\textquotesingle{}R (ohms)\textquotesingle{}: J\_294[\textquotesingle{}R (ohms)\textquotesingle{}],}
\CommentTok{\#                         \textquotesingle{}Vrms (V)\_294\textquotesingle{}: J\_294[\textquotesingle{}Vrms (V)\textquotesingle{}],}
\CommentTok{\#                         \textquotesingle{}Vrms (V)\_ln2\textquotesingle{}: J\_ln2[\textquotesingle{}Vrms (V)\textquotesingle{}],}
\CommentTok{\#                         \textquotesingle{}Vrms (V)\_diff\textquotesingle{}: J\_diff[\textquotesingle{}Vms (V) diff\textquotesingle{}].apply(np.abs).apply(unp.sqrt),}
\CommentTok{\#                     \})}
\CommentTok{\# big}
\end{Highlighting}
\end{Shaded}

\begin{verbatim}
/home/coyotedark/Documents/uw/phys/431/johnson-and-shot-noise-main/.venv/lib/python3.14/site-packages/pandas/core/internals/blocks.py:347: FutureWarning: AffineScalarFunc.__abs__() is deprecated. It will be removed in a future release.
  result = func(self.values, **kwargs)
\end{verbatim}

\begin{longtable}[]{@{}lllll@{}}
\toprule\noalign{}
& R (ohms) & nf\_294 & nf\_ln2 & nf\_diff \\
\midrule\noalign{}
\endhead
\bottomrule\noalign{}
\endlastfoot
0 & 40000.0 & 0.962+/-0.004 & 1.035+/-0.008 & 0.934+/-0.014 \\
1 & 20000.0 & 1.009+/-0.004 & 1.072+/-0.008 & 0.986+/-0.014 \\
2 & 15000.0 & 1.015+/-0.004 & 1.069+/-0.008 & 0.995+/-0.016 \\
3 & 9990.0 & 1.025+/-0.004 & 1.083+/-0.008 & 1.004+/-0.009 \\
4 & 7500.0 & 1.037+/-0.004 & 1.106+/-0.008 & 1.011+/-0.023 \\
5 & 4990.0 & 1.039+/-0.004 & 1.190+/-0.009 & 0.980+/-0.015 \\
6 & 2500.0 & 1.071+/-0.004 & 1.207+/-0.009 & 1.018+/-0.019 \\
7 & 1000.0 & 1.144+/-0.004 & 1.419+/-0.010 & 1.030+/-0.019 \\
8 & 0.0 & NaN & NaN & NaN \\
\end{longtable}

\subsection{Shot Noise Analysis}\label{shot-noise-analysis}

\begin{quote}
This is very similar to the Johnson noise analysis.
\end{quote}

\subsubsection{Read in the data}\label{read-in-the-data}

\begin{quote}
For data structure type ``1'', column names like ``0.1202mA'' need to
split at \texttt{m} to convert the current labels into currents.
\end{quote}

For the Shot Noise data, due to the battery constraints, we opted for
the automated collection of averages. This, however, means we do not
have the standard deviation. This means uncertainties will not be
reported.

\begin{Shaded}
\begin{Highlighting}[]
\NormalTok{shot }\OperatorTok{=}\NormalTok{ pd.read\_csv(}\StringTok{\textquotesingle{}./rawData/Johnson data {-} shot.csv\textquotesingle{}}\NormalTok{)}
\NormalTok{shot }\OperatorTok{=}\NormalTok{ shot.rename(columns}\OperatorTok{=}\NormalTok{\{}\StringTok{\textquotesingle{}Emission Current\textquotesingle{}}\NormalTok{ : }\StringTok{\textquotesingle{}Emission Current (A)\textquotesingle{}}\NormalTok{, }\StringTok{\textquotesingle{}Avr\textquotesingle{}}\NormalTok{ : }\StringTok{\textquotesingle{}Average (Vrms)\textquotesingle{}}\NormalTok{\}, errors}\OperatorTok{=}\StringTok{\textquotesingle{}raise\textquotesingle{}}\NormalTok{)}
\NormalTok{shot[}\StringTok{\textquotesingle{}Emission Current (A)\textquotesingle{}}\NormalTok{] }\OperatorTok{=}\NormalTok{ shot[}\StringTok{"Emission Current (A)"}\NormalTok{] }\OperatorTok{/} \DecValTok{1000}


\NormalTok{shot}
\end{Highlighting}
\end{Shaded}

\begin{longtable}[]{@{}lll@{}}
\toprule\noalign{}
& Emission Current (A) & Average (Vrms) \\
\midrule\noalign{}
\endhead
\bottomrule\noalign{}
\endlastfoot
0 & 0.000000 & 0.004807 \\
1 & 0.000010 & 0.007450 \\
2 & 0.000011 & 0.007873 \\
3 & 0.000012 & 0.008101 \\
4 & 0.000013 & 0.008657 \\
5 & 0.000014 & 0.008493 \\
6 & 0.000015 & 0.008643 \\
7 & 0.000020 & 0.009940 \\
8 & 0.000030 & 0.011280 \\
9 & 0.000040 & 0.012660 \\
10 & 0.000050 & 0.013750 \\
11 & 0.000060 & 0.014880 \\
12 & 0.000070 & 0.015910 \\
13 & 0.000080 & 0.017140 \\
14 & 0.000090 & 0.017990 \\
15 & 0.000100 & 0.018810 \\
16 & 0.000110 & 0.019710 \\
17 & 0.000120 & 0.020540 \\
18 & 0.000130 & 0.021360 \\
19 & 0.000140 & 0.022140 \\
20 & 0.000150 & 0.022680 \\
\end{longtable}

\subsubsection{Plot the raw data}\label{plot-the-raw-data}

\begin{Shaded}
\begin{Highlighting}[]
\CommentTok{\#\# Plot it}

\NormalTok{fig }\OperatorTok{=}\NormalTok{ shot.plot.scatter(}\StringTok{\textquotesingle{}Emission Current (A)\textquotesingle{}}\NormalTok{, }\StringTok{\textquotesingle{}Average (Vrms)\textquotesingle{}}\NormalTok{, marker}\OperatorTok{=}\StringTok{\textquotesingle{}d\textquotesingle{}}\NormalTok{)}
\NormalTok{ax }\OperatorTok{=}\NormalTok{ fig.axes}
\end{Highlighting}
\end{Shaded}

\pandocbounded{\includegraphics[keepaspectratio]{Noise Analysis (template, 2024)_files/figure-latex/cell-31-output-1.png}}

\subsubsection{\texorpdfstring{Calculate
\(V^2_{rms}\)}{Calculate V\^{}2\_\{rms\}}}\label{calculate-v2_rms}

\begin{Shaded}
\begin{Highlighting}[]
\CommentTok{\#\# transform the data, like you did with Johnson noise}
\NormalTok{shot[}\StringTok{\textquotesingle{}Vms\textquotesingle{}}\NormalTok{] }\OperatorTok{=}\NormalTok{ shot[}\StringTok{"Average (Vrms)"}\NormalTok{]}\OperatorTok{**}\DecValTok{2}
\NormalTok{shot}
\end{Highlighting}
\end{Shaded}

\begin{longtable}[]{@{}llll@{}}
\toprule\noalign{}
& Emission Current (A) & Average (Vrms) & Vms \\
\midrule\noalign{}
\endhead
\bottomrule\noalign{}
\endlastfoot
0 & 0.000000 & 0.004807 & 0.000023 \\
1 & 0.000010 & 0.007450 & 0.000056 \\
2 & 0.000011 & 0.007873 & 0.000062 \\
3 & 0.000012 & 0.008101 & 0.000066 \\
4 & 0.000013 & 0.008657 & 0.000075 \\
5 & 0.000014 & 0.008493 & 0.000072 \\
6 & 0.000015 & 0.008643 & 0.000075 \\
7 & 0.000020 & 0.009940 & 0.000099 \\
8 & 0.000030 & 0.011280 & 0.000127 \\
9 & 0.000040 & 0.012660 & 0.000160 \\
10 & 0.000050 & 0.013750 & 0.000189 \\
11 & 0.000060 & 0.014880 & 0.000221 \\
12 & 0.000070 & 0.015910 & 0.000253 \\
13 & 0.000080 & 0.017140 & 0.000294 \\
14 & 0.000090 & 0.017990 & 0.000324 \\
15 & 0.000100 & 0.018810 & 0.000354 \\
16 & 0.000110 & 0.019710 & 0.000388 \\
17 & 0.000120 & 0.020540 & 0.000422 \\
18 & 0.000130 & 0.021360 & 0.000456 \\
19 & 0.000140 & 0.022140 & 0.000490 \\
20 & 0.000150 & 0.022680 & 0.000514 \\
\end{longtable}

\begin{Shaded}
\begin{Highlighting}[]
\NormalTok{shot.plot.scatter(}\StringTok{\textquotesingle{}Emission Current (A)\textquotesingle{}}\NormalTok{, }\StringTok{\textquotesingle{}Vms\textquotesingle{}}\NormalTok{, marker}\OperatorTok{=}\StringTok{\textquotesingle{}d\textquotesingle{}}\NormalTok{)}
\end{Highlighting}
\end{Shaded}

\pandocbounded{\includegraphics[keepaspectratio]{Noise Analysis (template, 2024)_files/figure-latex/cell-33-output-1.png}}

\subsubsection{Then fit it and plot it}\label{then-fit-it-and-plot-it}

\textbf{Note:} Shot nose data may not be ``pure'' in that you will see a
notable deviation from the expected behavior. The data may be affected
by \(1/f\) noise in the vacuum diode that gets worse with higher
emission current. This effect is reduced in the newer shot noise
apparatus that uses a different vacuum diode. If you see a notable curve
in your measured voltage, you may try a couple of work-arounds to obtain
the linear part of the noise-squared vs emission current:

\begin{enumerate}
\def\labelenumi{\arabic{enumi}.}
\tightlist
\item
  Select a portion of the data to fit, where the \(1/f\) problem is
  less, near the low-emission current end of the data set.
\item
  Make a ploynomial fit and look at the linear term.
\end{enumerate}

You should try a couple of options and compare your results with your
partners. You only need to do this if you see the \(1/f\) effect.

\begin{Shaded}
\begin{Highlighting}[]
\CommentTok{\#\# First the fit. Try the whole data set first.}
\CommentTok{\# The data seems fairly linear, lets try fitting first.}
\NormalTok{shotparams }\OperatorTok{=}\NormalTok{ line\_fit\_and\_plot(shot[}\StringTok{\textquotesingle{}Emission Current (A)\textquotesingle{}}\NormalTok{], shot[}\StringTok{\textquotesingle{}Vms\textquotesingle{}}\NormalTok{])}

\end{Highlighting}
\end{Shaded}

\begin{verbatim}
[[Model]]
    Model(linear)
[[Fit Statistics]]
    # fitting method   = leastsq
    # function evals   = 4
    # data points      = 21
    # variables        = 2
    chi-square         = 2.0107e-10
    reduced chi-square = 1.0583e-11
    Akaike info crit   = -528.809393
    Bayesian info crit = -526.720348
    R-squared          = 0.99961758
[[Variables]]
    slope:      3.28674301 +/- 0.01474829 (0.45%) (init = 3.286743)
    intercept:  2.6791e-05 +/- 1.1372e-06 (4.24%) (init = 2.679105e-05)
\end{verbatim}

\pandocbounded{\includegraphics[keepaspectratio]{Noise Analysis (template, 2024)_files/figure-latex/cell-34-output-2.png}}

\begin{Shaded}
\begin{Highlighting}[]

\CommentTok{\#\# and the slope of the data is:}
\NormalTok{shot\_slope\_inclusive }\OperatorTok{=}\NormalTok{ get\_uslope(shotparams)}
\NormalTok{shot\_slope\_inclusive}
\end{Highlighting}
\end{Shaded}

\begin{verbatim}
3.28674300654674+/-0.014748286981550951
\end{verbatim}

\begin{Shaded}
\begin{Highlighting}[]
\CommentTok{\# Next, lets only get the data we collected more of.}
\CommentTok{\# We collected extra data from .01 {-} .015, so lets first split the data}

\NormalTok{shot\_small }\OperatorTok{=}\NormalTok{ shot.iloc[}\DecValTok{0}\NormalTok{:}\DecValTok{7}\NormalTok{]}
\NormalTok{shot\_small\_params }\OperatorTok{=}\NormalTok{ line\_fit\_and\_plot(shot\_small[}\StringTok{\textquotesingle{}Emission Current (A)\textquotesingle{}}\NormalTok{], shot\_small[}\StringTok{\textquotesingle{}Vms\textquotesingle{}}\NormalTok{])}
\NormalTok{shot\_small\_slope }\OperatorTok{=}\NormalTok{ get\_uslope(shot\_small\_params)}
\NormalTok{shot\_small\_slope}


\end{Highlighting}
\end{Shaded}

\begin{verbatim}
[[Model]]
    Model(linear)
[[Fit Statistics]]
    # fitting method   = leastsq
    # function evals   = 4
    # data points      = 7
    # variables        = 2
    chi-square         = 4.4992e-11
    reduced chi-square = 8.9984e-12
    Akaike info crit   = -176.393090
    Bayesian info crit = -176.501270
    R-squared          = 0.97744034
[[Variables]]
    slope:      3.58792818 +/- 0.24376977 (6.79%) (init = 3.587928)
    intercept:  2.2700e-05 +/- 2.8473e-06 (12.54%) (init = 2.270023e-05)
\end{verbatim}

\begin{verbatim}
3.5879281820754727+/-0.2437697702576206
\end{verbatim}

\pandocbounded{\includegraphics[keepaspectratio]{Noise Analysis (template, 2024)_files/figure-latex/cell-36-output-3.png}}

\subsubsection{Calculate Electron
Charge}\label{calculate-electron-charge}

\begin{quote}
Use the fit results, propagate the uncertainty, and find a value for
\(e\).
\end{quote}

The equation given is: \[I_{shot} = \sqrt{2eI_{emi}B}\]

The slope we found is \(V^2/I\), so lets use that to find e.

\begin{Shaded}
\begin{Highlighting}[]
\CommentTok{\#\# Calculate e with uncertainty and print it (with units) }
\CommentTok{\#\# Compare with the accepted value}

\CommentTok{\# You will need the correct sensing resistance in the shot noise box:}
\CommentTok{\# Older box:}
\CommentTok{\# R\_load = unc.ufloat(4976,1) \# Load resistance of shot noise box in ohms }
\CommentTok{\# Newer box:}
\NormalTok{R\_load }\OperatorTok{=}\NormalTok{ unc.ufloat(}\FloatTok{10000.0}\NormalTok{,}\DecValTok{10}\NormalTok{)}

\CommentTok{\# e\_1 = .5 * (1000 * R\_load ** 2) ** {-}1 * shot\_slope\_inclusive \# This is mA}
\NormalTok{e\_1 }\OperatorTok{=}\NormalTok{ (}\DecValTok{1} \OperatorTok{/}\NormalTok{ (}\DecValTok{2} \OperatorTok{*}\NormalTok{ (G }\OperatorTok{**} \DecValTok{2}\NormalTok{) }\OperatorTok{*} \DecValTok{1000} \OperatorTok{*}\NormalTok{ (R\_load }\OperatorTok{**} \DecValTok{2}\NormalTok{))) }\OperatorTok{*}\NormalTok{ shot\_slope\_inclusive}

\NormalTok{e\_2 }\OperatorTok{=}\NormalTok{ (}\DecValTok{1} \OperatorTok{/}\NormalTok{ (}\DecValTok{2} \OperatorTok{*}\NormalTok{ (G }\OperatorTok{**} \DecValTok{2}\NormalTok{) }\OperatorTok{*} \DecValTok{1000} \OperatorTok{*}\NormalTok{ (R\_load }\OperatorTok{**} \DecValTok{2}\NormalTok{))) }\OperatorTok{*}\NormalTok{ shot\_small\_slope}

\CommentTok{\# We have th}



\CommentTok{\# Calculate the result, and propagate the uncertainty.}

\CommentTok{\# Use whatever you need below}
\BuiltInTok{print}\NormalTok{(}\StringTok{\textquotesingle{}}\CharTok{\textbackslash{}n}\StringTok{Electron charge from whole data set = \{:.2uP\} C\textquotesingle{}}\NormalTok{.}\BuiltInTok{format}\NormalTok{(e\_1))}
\BuiltInTok{print}\NormalTok{(}\StringTok{\textquotesingle{}Electron charge from partial data set = \{:.2uP\} C\textquotesingle{}}\NormalTok{.}\BuiltInTok{format}\NormalTok{(e\_2))}
\CommentTok{\# print(\textquotesingle{}Electron charge from quadratic fit = \{:.2uP\} C\textquotesingle{}.format(e\_3))}
\BuiltInTok{print}\NormalTok{(}\StringTok{\textquotesingle{}}\CharTok{\textbackslash{}n}\StringTok{Accepted value = }\SpecialCharTok{\{:.4g\}}\StringTok{ C\textquotesingle{}}\NormalTok{.}\BuiltInTok{format}\NormalTok{(const.e))}
\end{Highlighting}
\end{Shaded}

\begin{verbatim}

Electron charge from whole data set = (1.604±0.014)×10⁻¹⁹ C
Electron charge from partial data set = (1.75±0.12)×10⁻¹⁹ C

Accepted value = 1.602e-19 C
\end{verbatim}




\end{document}
